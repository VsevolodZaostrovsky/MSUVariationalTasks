\documentclass[a4paper]{article}

%Russian-specific packages
%--------------------------------------
\usepackage[T2A]{fontenc}
\usepackage[utf8]{inputenc}
\usepackage[english, russian]{babel}
%for search in russian
\usepackage{cmap}
%--------------------------------------

%Math-specific packages
%--------------------------------------
\usepackage{amsmath}
\usepackage{amssymb}
\usepackage{hyperref}

%Format-specific packages
%--------------------------------------
\usepackage[left=1cm,
            right=1cm,
            top=1cm,
            bottom=1cm,
            bindingoffset=0cm]{geometry}
%--------------------------------------

% for theorems, lemmas and definitions
%--------------------------------------
\usepackage{amsthm}

\newtheorem{theorem}{Теорема}

\theoremstyle{definition}
\newtheorem {definition}{Опрeделение}
\newtheorem {task}{Задача}

%--------------------------------------

%Roman enum items
\usepackage{enumerate}

% For graphics
%--------------------------------------
\usepackage{tikz}
\usetikzlibrary{
  % for faster compilation
  external
  % for cool arrows
  , arrows.meta
  % for angles
  , angles
  , quotes
  , babel
}
\tikzsetexternalprefix{tasks/}
\tikzexternalize

%--------------------------------------

% My commands
%--------------------------------------

\DeclareMathOperator{\sgn}{sgn}

\def\const{ \mathrm{const} }
\def\eps{ \varepsilon }
\def\Eps{ \mathcal{E} }

\def\R{ \mathbb{R} }
\def\Z{ \mathbb{Z} }
\def\C{ \mathbb{C} }
\def\E{ \mathrm{E} }
\def\D{ \mathrm{D} }
\def\P{ \mathrm{P} }

\def\littleO{ \overline{\overline{o}} }
\def\bigO{ \underline{\underline{\mathcal{O}}} }

\newcommand*{\norm}[1]{\left\lVert#1\right\rVert}
\newcommand*{\abs}[1]{\left\lvert#1\right\rvert}

\date{Осенний семестр 2023}
\title{Курс <<Вариационное исчисление и оптимальное управление>>}

% suppress page count
\pagestyle{empty}

\includeonly{
  tasks/task01/task,
  tasks/task02/task,
  tasks/task03/task,
  tasks/task04/task,
  tasks/task05/task,
  tasks/task06/task,
  tasks/task07/task,
  tasks/task08/task,
  tasks/task09/task,
  tasks/task10/task,
  tasks/task11/task,
  tasks/task12/task,
  tasks/task13/task,
  tasks/task14/task,
  tasks/task15/task,
  tasks/task16/task,
  tasks/task17/task,
  tasks/task18/task,
  tasks/task19/task,
  tasks/task20/task,
  tasks/task21/task,
  tasks/task22/task,
  tasks/task23/task,
  tasks/task24/task,
  tasks/task25/task,
  tasks/task26/task,
  tasks/task27/task,
  tasks/task28/task
}

\begin{document}

\begin{titlepage}
  \maketitle
  
  \begin{abstract}
    В документе собраны решения задач к экзамену по вариационному исчислению и оптимальному управлению
    осеннего семестра 2023 года. Лектор: Васильева А. А. 
    Задачи подготовлены дилетантами и могут содержать ошибки и опечатки. Исходники, материалы
    и информацию по участию в дополнении теха можно найти
    \href{https://github.com/VsevolodZaostrovsky/MSUVariationalTasks}{тут}.
  \end{abstract}
  
\end{titlepage}


\begin{task}
\textbf{Предварительный материал из лекции (Гармонический осциллятор):}\\
Рассмотрим задачу:
$$
\mathcal{L}(x):=\int_{0}^{T_{0}}\left(\dot{x}^{2}-x^{2}\right) d t \rightarrow \mathrm{inf}, \quad x(0)=x\left(T_{0}\right)=0
$$

Тогда $L_{\dot{x}}=2 \dot{x}, L_{x}=-2 x$; уравнение Эйлера имеет вид $-\frac{d}{d t}(2 \dot{x})-2 x=0$, т.e. $\ddot{x}+x=0$.

Заметим, что $\hat{x}=0$ является допустимой экстремалью. Выясним, является ли она точкой локального или глобального минимума. Для этого используем следующий прием.

Пусть $\omega \in C^{1}\left[0, T_{0}\right]$. Тогда $$\int_{0}^{T_{0}} \left(\dot{\omega} x^{2}+2 \omega x \dot{x} \right)d t=\int_{0}^{T_{0}} \frac{d}{d t}\left(\omega x^{2}\right) d t=\left.\omega x^{2}\right|_{0} ^{T_{0}}=0\, \text{, если } x \in C_{0,0}^{1}\left[0, T_{0}\right].$$ \\
Значит, можно добавить этот ноль к интегралу:
$$
\int_{0}^{T_{0}}\left(\dot{x}^{2}-x^{2}\right) d t=\int_{0}^{T_{0}}\left(\dot{x}^{2}-x^{2}-\dot{\omega} x^{2}-2 \omega x \dot{x}\right) d t
$$

Подберем $\omega$ так, чтобы $\dot{x}^{2}-x^{2}-\dot{\omega} x^{2}-2 \omega x \dot{x}$ было полным квадратом: $\dot{x}^{2}-x^{2}-\dot{\omega} x^{2}-2 \omega x \dot{x}=(\dot{x}-\omega x)^{2}$, т.е. $-1-\dot{\omega}=\omega^{2}$. (Тогда $\int_{0}^{T_{0}}\left(\dot{x}^{2}-x^{2}\right) d t=$ $\left.\int_{0}^{T_{0}}(\dot{x}-\omega x)^{2} d t \geq 0.\right)$ Получаем, что $\omega=\operatorname{ctg}\left(t-t_{*}\right)$.
\\
\textbf{Задача}\\

1) Пусть $T_{0}>\pi,\ x(t)=c \sin \frac{\pi t}{T_{0}}$. Показать, что $\mathcal{L}(x)<0$ при $c \neq 0$. Почему проведенные выше рассуждения не проходят при $T_{0}>\pi$ и проходят при $T_{0}<\pi$ ?\\
2) Показать, что $\int_{0}^{\pi}\left(\dot{x}^{2}-x^{2}\right) d t=\int_{0}^{\pi}(\dot{x}-x \cdot \operatorname{ctg} t)^{2} d t \geqslant 0$.

\textbf{Решение.} 1) Вычисляем:
$$
\begin{gathered}
\int_{0}^{T_{0}}\left(\dot{x}^{2}-x^{2}\right) d t=c^{2} \int_{0}^{T_{0}}\left(\frac{\pi^{2}}{T_{0}} \cos ^{2} \frac{\pi t}{T_{0}}-\sin ^{2} \frac{\pi t}{T_{0}}\right) d t= \\
=\frac{c^{2}}{2}\left(\frac{\pi^{2}}{T_{0}^{2}} \int_{0}^{T_{0}}\left(1+\cos \frac{2 \pi t}{T_{0}}\right) d t-\int_{0}^{T_{0}}\left(1-\cos \frac{2 \pi t}{T_{0}}\right) d t\right)=\frac{c^{2} T_{0}}{2}\left(\frac{\pi^{2}}{T_{0}^{2}}-1\right)<0 .
\end{gathered}
$$

Напомним, что мы подбирали гладкую функцию $\omega$ так, чтобы

$$
\int_{0}^{T_{0}}\left(\dot{x}^{2}-x^{2}\right) d t=\int_{0}^{T_{0}}(\dot{x}-\omega x)^{2} d t
$$

при этом $\omega$ была решением дифференциального уравнения $\dot{\omega}=-1-\omega^{2}$. Значит, $\omega(t)=$ $\operatorname{ctg}(t-a)$.

Если $T_{0}<\pi$, то можно подобрать $a$ (в данном случае подойдет $a = 0$) так, чтобы $\operatorname{ctg}(t-a)$ была гладкой на $\left[0, T_{0}\right]$. Если $T_{0}>\pi$, то для любого $a$ функция $\omega$ будет иметь точку разрыва в интервале $\left(0, T_{0}\right)$, т.к. $ctg$ гладко определен на $(\pi n ,\ \pi n + \pi)$ и имеет разрывы в точках $\pi n$.

2) Преобразуем правую часть условия:
$$
\begin{gathered}
0 \leq\int_{0}^{\pi}(\dot{x}-x \cdot \operatorname{ctg} t)^{2} d t=\int_{0}^{\pi}\left(\dot{x}^{2}-2 x \dot{x} \operatorname{ctg} t+x^{2} \operatorname{ctg}^{2} t\right) d t= 
\end{gathered}
$$\\
интегрируем среднее слагаемое по частям и используем $\dot{ctg}+ctg^{2}=-1$:
$$
\begin{gathered}
=\int_{0}^{\pi}\left(\dot{x}^{2}+x^{2}(\operatorname{ctg} t)^{\prime}+x^{2} \operatorname{ctg}^{2} t\right) d t+\left.x^{2}(t) \operatorname{ctg} t\right|_{0} ^{\pi}=\int_{0}^{\pi}\left(\dot{x}^{2}-x^{2}\right) d t+\left.x^{2}(t) \operatorname{ctg} t\right|_{0} ^{\pi} .
\end{gathered}
$$

Так как $x \in C^{1}[0, \pi]$ и $x(0)=0$, то $x(t)=O(t)$ в окрестности нуля; так как $\operatorname{ctg} t=O(1 / t)$ в окрестности нуля, то $x^{2}(t) \operatorname{ctg} t=O(t) \underset{t \rightarrow 0}{\rightarrow} 0$. Аналогично $x^{2}(t) \operatorname{ctg} t=O(\pi-t) \underset{t \rightarrow \pi}{\rightarrow} 0$. Значит, $\left.x^{2}(t) \operatorname{ctg} t\right|_{0} ^{\pi}=0$ и равенство доказано.

\end{task}

\begin{task}
    Доказать, что в задаче
    
    $$
    \int_{0}^{1} t^{1 / 2} \dot{x}^{2} d t \rightarrow \inf , \quad x(0)=0, \quad x(1)=1
    $$
    
    точки локального минимума в пространстве $C^{1}[0,1]$ не существует; найти точку глобального минимума для той же задачи в пространстве
    
    $$
    W=\left\{f \in A C[0,1]: \int_{0}^{1} t^{1 / 2} \dot{x}^{2} d t<\infty\right\}
    $$
    
    \textbf{Решение.} Напишем уравнение Эйлера: $\frac{d}{d t}\left(2 t^{1 / 2} \dot{x}\right)=0$, откуда $t^{1 / 2} \dot{x}=c$. Значит, $x=$ $2 c t^{1 / 2}+b$. Подставляя граничные условия, получаем $x=t^{1 / 2} \notin C^{1}[0,1]$.
    
    Пусть $h \in W, h(0)=h(1)=0$. Тогда
    
    $$
    \begin{gathered}
    \int_{0}^{1} t^{1 / 2}(\dot{x}+\dot{h})^{2} d t-\int_{0}^{1} t^{1 / 2} \dot{x}^{2} d t=\int_{0}^{1} t^{1 / 2}\left(2 \dot{x} \dot{h}+\dot{h}^{2}\right) d t=\int_{0}^{1} t^{1 / 2}\left(t^{-1 / 2} \dot{h}+\dot{h}^{2}\right) d t= \\
    =\int_{0}^{1} \dot{h} d t+\int_{0}^{1} t^{1 / 2} \dot{h}^{2} d t=\int_{0}^{1} t^{1 / 2} \dot{h}^{2} d t \geqslant 0 .
    \end{gathered}
    $$
\end{task}
    
\begin{task}
    Доказать, что в задаче
    $$
    \int_0^1\left(1-\dot{x}^2\right)^2 d t \rightarrow \text { inf, } \quad x(0)=0, \quad x(1)=0
    $$
    
    точки локального минимума в пространстве $C^1[0,1]$ не существует, при этом точная нижняя грань функционала равна 0.
    
    \textbf{Решение.} Заметим, что $\int_0^1\left(1-\dot{x}^2\right)^2 d t \geqslant 0$. Если $\int_0^1\left(1-\dot{x}^2\right)^2 d t=0$, то $\dot{x}^2(t) \equiv 1$, откуда $\dot{x}(t)= \pm 1$ для любого $t$. Так как $\dot{x}$ непрерывна, то $\dot{x} \equiv 1$ или $\dot{x} \equiv-1$. Получаем противоречие с граничными условиями. Значит, нулевое значение не достигается.
    
    Теперь покажем, что для любого $\varepsilon>0$ существует допустимая функция $x \in C^1[0,1]$ такая, что $\int_0^1\left(1-\dot{x}^2\right)^2 d t \leqslant \varepsilon$. Положим
    $$
    z(t)=\left\{\begin{array}{l}
    1, \quad 0 \leqslant t \leqslant \frac{1}{2}-\delta \\
    \frac{1}{\delta}\left(\frac{1}{2}-t\right), \quad \frac{1}{2}-\delta \leqslant t \leqslant \frac{1}{2}+\delta, \\
    -1, \quad \frac{1}{2}+\delta \leqslant t \leqslant 1
    \end{array}\right.
    $$
    $x(t)=\int_0^t z(s) d s$. Тогда $x \in C^1[0,1], x(0)=x(1)=0$. При этом $|\dot{x}| \leqslant 1$. Значит,
    $$
    \int_0^1\left(1-\dot{x}^2\right)^2 d t=\int_{\frac{1}{2}-\delta}^{\frac{1}{2}+\delta}\left(1-\dot{x}^2\right)^2 d t \leqslant 2 \delta .
    $$
    
    Значит, достаточно взять $\delta=\frac{\varepsilon}{2}$.
    \end{task}
\begin{task}
(задача о геодезических на плоскости Лобачевского.) Найти допустимые экстремали в задаче

$$
\int_{t_{0}}^{t_{1}} \frac{\sqrt{1+\dot{x}^{2}}}{x} d t \rightarrow \operatorname{extr}, x\left(t_{0}\right)=x_{0}, \quad x\left(t_{1}\right)=x_{1}, \quad x>0
$$

\textbf{Решение.} 
Имеем $L_{\dot{x}}=\frac{\dot{x}}{x \sqrt{1+\dot{x}^{2}}}, L_{\dot{x} \dot{x}}=\frac{1}{x\left(1+\dot{x}^{2}\right)^{3 / 2}}>0$. Значит, $\hat{x} \in$ $C^{2}\left[t_{0}, t_{1}\right]$ и $\dot{\hat{x}}(t) L_{\dot{x}}(\hat{x}(t), \dot{\hat{x}}(t))-L(\hat{x}(t), \dot{\hat{x}}(t))=$ const.

Проверим, что допустимая экстремаль не может обращаться в константу ни на каком невырожденном интервале. Это видно из уравнения Эйлера:

$$
-\frac{d}{d t} \frac{\dot{x}}{x \sqrt{1+\dot{x}^{2}}}-\frac{\sqrt{1+\dot{x}^{2}}}{x^{2}}=0
$$

(тогда бы получилось равенство $1 / \hat{x}^{2}(t) \equiv 0$ ).

Из уравнения $\dot{x} L_{\dot{x}}-L=$ const получаем

$$
\dot{x} \cdot \frac{\dot{x}}{x \sqrt{1+\dot{x}^{2}}}-\frac{\sqrt{1+\dot{x}^{2}}}{x}=\text { const. }
$$

Значит, $\frac{1}{x \sqrt{1+\dot{x}^{2}}}=$ const. Получаем $1+\dot{x}^{2}=\frac{c^{2}}{x^{2}}$, или $\dot{x}= \pm \sqrt{1-\frac{c^{2}}{x^{2}}}$. На промежутках, где $\dot{x} \neq 0$, решаем это дифференциальное уравнение и получаем

$$
t-a= \pm \int \frac{d x}{\sqrt{1-\frac{c^{2}}{x^{2}}}}= \pm \frac{1}{2} \int \frac{d x^{2}}{\sqrt{x^{2}-c^{2}}}= \pm \sqrt{x^{2}-c^{2}}
$$

Возводим в квадрат и получаем $x^{2}+(t-a)^{2}=c^{2}$. Это уравнение окружности с центром на горизонтальной оси.

Если в какой-то точке $\dot{x}$ обращается в 0 , то условия теоремы единственности нарушаются, но всё равно экстремаль задается уравнением окружности (склеивается из двух дуг окружностей; в силу гладкости обе дуги принадлежат одной и той же окружности; горизонтальных "вставок" быть не может, т.к. экстремаль не равна константе на интервалах).

Итак, геодезические - дуги окружности с центром на горизонтальной оси.

Утверждается, что найденная допустимая экстремаль будет точкой глобального минимума. В самом деле, $L_{\dot{x} \dot{x}}>0$ при $x>0$, так что $L$ выпукла по $\dot{x}$.

\end{task}
\begin{task}
    Найти допустимые экстремали в задаче
    $$
    \int_{t_0}^{t_1}\frac{\sqrt{1+\dot{x}^2}}{\sqrt{x}} dt \rightarrow \max ,\quad x(t_0)=x_0,\quad x(t_1)=x_1,\quad x>0
    $$
    
    \textbf{Решение.}
    $L$ явно не зависит от $t$. Если $\hat{x}$ - экстремаль, $\hat{x}\in C^2$, то
    $$
    \dot{x}L_{\dot{x}}-L = const
    $$
    
    Так как $L\in C^2$, достаточно доказать, что $L_{\dot{x}\dot{x}}\neq 0$
    
    $L_{\dot{x}\dot{x}}=\frac{1}{\sqrt{x}(1+\dot{x}^2)^{3/2}}$
    
    $\hat{x}\in C^2, L\in C^2 $ удовлетворяют $\dot{x}L_{\dot{x}} = const$ 
    
    $\hat{x}$ не равна константе ни на каком интервале $\Longrightarrow\hat{x}$ удовлетворяет уравнению Эйлера
    
    $\hat{x}=c$ на интервале
    
    $-\frac{d}{dt}\frac{\dot{x}}{\sqrt{x}\sqrt{1+\dot{x}^2}}+\frac{1}{2}x^{-3/2}\sqrt{1+\dot{x}^2}=0$
    
    Первое слагаемое ноль, второе - ненулевая константа. Противоречие.
    
    $$
    \dot{x}L_{\dot{x}}-L = const
    $$
    $$
    \dot{x}\frac{\dot{x}}{\sqrt{x}\sqrt{1+\dot{x}^2}} - \frac{\sqrt{1+\dot{x}^2}}{\sqrt{x}}=const
    $$
    $$
    x(1+\dot{x}^2)=const
    $$
    $$
    \dot{x}^2=\frac{c}{x}-1
    $$
    $$
    \dot{x}=\pm\sqrt{\frac{c}{x}-1}
    $$
    $$
    x = c\sin^2{\frac{\tau}{2}}=\frac{c}{2}(1-\cos{\tau})
    $$
    $$
    t+a= \pm\int\frac{dx}{\sqrt{\frac{c}{x}-1}} = \pm\int\frac{2c\sin{\frac{\tau}{2}}\cos{\frac{\tau}{2}}d\tau}{\sqrt{\frac{1}{\sin^2{\frac{\tau}{2}}}-1}} = \pm c\int\sin{\frac{\tau}{2}}\cos{\frac{\tau}{2}}\abs{\tan{\frac{\tau}{2}}}d\tau
    $$
    $$
    c\int\sin^2{\frac{\tau}{2}}d\tau = \frac{c}{2}\int(1-\cos{\tau})d\tau = \frac{c}{2}(\tau-\sin{\tau})
    $$
    
\end{task}


\begin{task} \label{task6}
    Найти допустимые экстремали в задаче
    \[  \int_{-T_{0}}^{T_{0}} x\sqrt{1 + \dot{x}} \,dt \rightarrow extr\text{, }x(-T_{0}) = x(T_{0}) = \xi\text{, } x > 0 \]
    В зависимости от $\xi > 0$ установить, сколько может быть допустимых экстремалей. \\
    \textbf{Решение.} \\
    \[L_{\dot{x}} = \frac{x\dot{x}}{\sqrt{1 + \dot{x}^2}}\]
    \[L_{\dot{x}\dot{x}} = {\frac{x}{(1 + \dot{x}^{2})}^ {3/2}} > 0,\]
    Решение уравнения Эйлера $\hat{x} \in C^2[-T_{0}, T_{0}]$ и $\hat{x}(t)\dot{L_{x}}(\hat{x}(t),
        \dot{\hat{x}}(t)) - L(\hat{x}(t),
        \dot{\hat{x}}(t)) = const$.\\
    Проверим, что допустимая экстремаль не может обращаться в константу ни на каком невырожденном отрезке. Это видно из уравнения Эйлера:
    \[-\frac{d}{dt} \frac{x\dot{x}}{\sqrt{1 + \dot{x}^2}} + \sqrt{1 + \dot{x}} = 0\]
    Решений нет. \\
    Из уравнения $\dot{x}L_{\dot{x}} - L = const$ получаем: \\
    \[ \dot{x} \frac{x\dot{x}}{\sqrt{1 + \dot{x}^2}} - x\sqrt{1 + \dot{x}} = const\]
    \[\frac{x}{\sqrt{1 + \dot{x}^2}} = const = C \Rightarrow 1 + \dot{x}^2 = \frac{x^2}{C^2} \Rightarrow \dot{x} = \pm \sqrt{\frac{x^2}{C^2} - 1} \Rightarrow dt = \pm \frac{dx}{\sqrt{\frac{x^2}{C^2} - 1}}\]
    \[t + a = \pm \int_{}^{}\frac{1}{\sqrt{\frac{x^2}{C^2} - 1}} \,dx = \left| x = Cch(\tau) \right| = \pm \int_{}^{}\frac{C sh(\tau)}{|sh(\tau)|} \,d\tau \]
    \[t + a = \pm C\tau \Rightarrow \frac{t+a}{C} = \pm \tau \Rightarrow ch\left(\frac{t+a}{C}\right) = ch(\tau) = \frac{x}{C} \Rightarrow x = Cch\left(\frac{t+a}{C}\right)\]
    \[x(-T_{0}) = x(T_{0}) \Rightarrow x = Cch\left(\frac{t}{C}\right)\]
    $C$ --- решение уравнения $Cch\left(\frac{T_{0}}{C}\right) = \xi$. Пусть $b = \frac{1}{C}$, тогда $ch(T_{0}b) = \xi b$. $ch(T_{0}b)$ --- выпуклая вниз функция, а $\xi b$ --- линейная $\Rightarrow$ может быть 0, 1 или 2 решениия.
    \[\xi_{*} > 0 \text{ --- случай касания.}\]
    \[\xi > \xi_{*} \text{ --- два решения.}\]
    \[\xi < \xi_{*} \text{ --- нет решений.}\]
    $ch(T_{0}b) = \xi_{*}b$, $T_{0}sh(T_{0}b) = \xi_{*}$. Исключаем $\xi_{*}$.
    \[ch(T_{0}b) = T_{0}bsh(T_{0}b)\]
    Отсюда находим $b$, а потом $\xi_{*}$.
\end{task}
\begin{task}
    Пусть $F: \mathbb{R}^2 \rightarrow \mathbb{R}$ задано равенством $F\left(x_1, x_2\right)=\sqrt[3]{x_1^2 x_2}$. Показать, что $F$ имеет вариацию по Лагранжу, но не дифференцируемо по Гато в нуле. \textbf{2)} Пусть $X$ -- бесконечномерное нормированное пространство, $F: X \rightarrow \mathbb{R}$ - линейный неограниченный функционал. Показать, что $F$ имеет вариацию по Лагранжу в нуле, но не дифференцируемо по Гато.
    
    \textbf{Решение.} \textbf{1)} Пусть $h = \left( h_1, h_2 \right)$. Тогда $F\left(t h_1, t h_2\right)=\sqrt[3]{\left(t h_1\right)^2 t h_2}=t \sqrt[3]{h_1^2 h_2}$. Значит,
    $$
    \frac{F\left(t h_1, t h_2\right)-F(0,0)}{t}=\sqrt[3]{h_1^2 h_2}, \quad F^{\prime}(0,0)\left[\left(h_1, h_2\right)\right]=\sqrt[3]{h_1^2 h_2}
    $$ 
    легко видеть, что это отображение нелинейно.\\
    \textbf{2)} Если функционал $F$ линеен, то $F(t h)-F(0)=t F(h)$; значит, $F^{\prime}(0)[h]=F(h)$. Это отображение линейно, но разрывно.
    
    \end{task}
\begin{document}

\textbf{Задача 8.}

Пусть $M=\left\{\left(x_{1}, x_{2}\right) \in \mathbb{R}^{2}: x_{1}>0, x_{2}=x_{1}^{2}\right\}, f: \mathbb{R}^{2} \rightarrow \mathbb{R}$,

$$
f\left(x_{1}, x_{2}\right)= \begin{cases}1, & \left(x_{1}, x_{2}\right) \in M \\ 0, & \left(x_{1}, x_{2}\right) \notin M\end{cases}
$$

Показать, что $f$ дифференцируемо по Гато, но не дифференцируемо по Фреше в т. $(0,0)$.\\

\textbf{Решение.} Вычислим вариацию по Лагранжу в нуле. Пусть $h \in \mathbb{R}^{2}$. Заметим, что прямая $\{t h: t \in \mathbb{R}\}$ пересекается с множеством $M$ не более, чем в одной точке. Значит, при малых $t$ выполнено $t h \notin M, F(t h)-F(0)=0$. Поэтому $F^{\prime}(0)[h]=0$. Это линейный непрерывный функционал. Значит, $F$ дифференцируемо по Гато в 0 . При этом $F$ в нуле разрывно и, следовательно, не дифференцируемо по Фреше.

\end{document}

\begin{task}
    Построить пример отображений $F:\mathbb{R}\rightarrow \mathbb{R}^2$, $G:\mathbb{R}^2\rightarrow \mathbb{R}$ таких, что $F$ дифференцируемо по Фреше в т. $0$, $G$ дифференцируемо по Гато в т. $(0,0)$, $F(0)=(0,0)$, при этом $G \circ F$ не имеет вариации по Лагранжу в т. 0.
    
    \textbf{Peшение.} 
    $$F:\mathbb{R}\rightarrow \mathbb{R}^2, \ \ \ x \rightarrow (x,x^2)$$
    $$G:\mathbb{R}^2\rightarrow \mathbb{R}, (x_1,x_2)\rightarrow$$ 
    \begin{equation*}
    G:\mathbb{R}^2\rightarrow \mathbb{R}, (x_1,x_2)\rightarrow
    \begin{cases}
        1, (x_1,x_2)\in M \\0, (x_1,x_2)\notin M
    \end{cases}
    \end{equation*}
    Где $M=\{ (x_1,x_2)|x_2=x_1,x_1>0\}$, уже знаем что G дифф. по Гато в т. (0,0)(но не по Фреше).Покажем, что $F$ дифф. по Фреше в т. 0.
    $$\underset{\lambda\rightarrow0}{lim}=\frac{F(\lambda h)-F (0)}{\lambda}=\underset{\lambda\rightarrow0}{lim}=\frac{(\lambda h,\lambda^2 h^2)-(0,0)}{\lambda}=(h,0)=F'(0)[h]$$
    $$F(\lambda h)=(\lambda h,\lambda^2 h^2)=(0,0)+(h,0)+((\lambda-1)h,\lambda^2 h^2)$$
    Последнее слогаемое это $\bar{o}(\|h \|)$.
    
    \noindent $\Rightarrow F$ дифф. по Фреше в 0.$G \circ F:\mathbb{R}\rightarrow\mathbb{R}$:
     \begin{equation*}
    x\rightarrow
    \begin{cases}
        1, x\neq0\\0, x=0
    \end{cases}
    \end{equation*}
    $$\underset{\lambda\rightarrow0}{lim}=\frac{G \circ F(\lambda h)-G \circ F(0)}{\lambda}=\underset{\lambda\rightarrow0}{lim} \left( \frac{1}{\lambda} \right) $$
    $\Rightarrow$ нет вариации по Лагранжу.
    \end{task}
\begin{task}
\textbf{Задача 10. }Привести пример функции $f: \mathbb{R} \rightarrow \mathbb{R}$, всюду дифференцируемой по Фреше, но не строго дифференцируемой в нуле.

\textbf{Доказательство:}
Рассмотрим функцию $f$:
\[
f(x)= \begin{cases}0, & x=0 \\ x^2 \sin \left(\frac{1}{x}\right), & x \neq 0\end{cases} 
\]

\begin{figure}[h!]
\centering
\includegraphics[width=0.99\linewidth]{image.png}
\end{figure}
 
$f$ дифференцируема по Фреше в $x \neq 0$ (по правилу Лейбница):
\[\begin{aligned}
&(x+h)^2 = x^2 +2xh+ \bar{o}(|h|)\\
& \sin{\frac{1}{x+h}} = \sin{\frac{1}{x}} + \frac{-1}{x^2} \cos{\frac{1}{x}}h + \bar{o}(|h|)
\end{aligned}\]
$|f(x)| \leq x^2 \implies f^{\prime}(0)=0$ и $f(x)-f(0)=f(x) =\bar{o}(|x|), |x| \rightarrow 0 $
то есть f диф. по Фреше в 0, тогда и на всем $\mathbb{R}$

Предположим противное: $f$ строго дифференцируемо в $0$.

По определению,  для любого $\varepsilon>0$ существует $\delta>0$ такое, что для любых $x_1, x_2 \in O_\delta\left(x_0\right)$ выполнено
\[\begin{aligned}
&|f\left(x_1\right)-f\left(x_2\right)-A\left(x_1-x_2\right)|\leq \varepsilon|x_1-x_2| \\
&f^{\prime}(0)=0 \implies |f\left(x_1\right)-f\left(x_2\right)|\leq \varepsilon|x_1-x_2|\leq 2\varepsilon \delta \\
& x\neq 0:  f^{\prime}(x)=2 x \sin \frac{1}{x}+x^2 \cos \frac{1}{x}\left(-1 \cdot \frac{1}{x^2}\right)=2 x \sin \left(\frac{1}{x}\right)-\cos \frac{1}{x} \nrightarrow 0, x \rightarrow 0 \\
& \text{потому что } \lim _{x \rightarrow 0} x \sin \frac{1}{x}=0 \text{ и } \lim _{x \rightarrow 0} \cos \frac{1}{x} \text{ не существует.} \\
& \text{Тогда }\exists \xi_n \rightarrow 0 \quad\left|f^{\prime}\left(\xi_n\right)\right| \geqslant c>0 \\
& f\left(\xi_n+h_n\right)-f\left(\xi_n\right)=f^{\prime}\left(\xi_n\right) h_n+\bar{o}\left(h_n\right) \text{ (по опр. диф. по Фреше)}\\
& h_n \longrightarrow 0 :\\
& \left|f\left(\xi_n+h_n\right)-f\left(\xi_n\right)\right| \geqslant C\left|h_n\right|
\end{aligned}\]
Получили противоречие. $f$ не является строго дифференцируемой в нуле.
\end{task}

\begin{task}
    Задача. (с лекций). 1) Если $F: X \rightarrow \mathbb{R}$, то существует $x \in\left[x_0, x_1\right]$ такое, что $F\left(x_1\right)-$ $\left.F\left(x_0\right)=F^{\prime}(x)\left[x_1-x_0\right] .2\right)$ Если $F: X \rightarrow Y, \operatorname{dim} Y>1$, то утверждение из п. 1 может быть неверным.
    
    \textbf{Решение.} 1) Положим $\varphi(t)=F\left((1-t) x_0+t x_1\right)$. Тогда $\varphi^{\prime}(t)=F^{\prime}\left((1-t) x_0+t x_1\right)\left[x_1-x_0\right]$. По теореме Лагранжа, существует $\tau \in(0,1)$ такое, что $\varphi(1)-\varphi(0)=\varphi^{\prime}(\tau)$. Значит,
    $$
    F\left(x_1\right)-F\left(x_0\right)=\varphi(1)-\varphi(0)=\varphi^{\prime}(\tau)=F^{\prime}\left((1-\tau) x_0+\tau x_1\right)\left[x_1-x_0\right] .
    $$
    2) Пусть $F: \mathbb{R} \rightarrow \mathbb{R}^2, F(t)=(\cos t, \sin t)$. Тогда $F(2 \pi)-F(0)=(0,0), F^{\prime}(t)=$ $(-\sin t, \cos t)$; значит, если $F(2 \pi)-F(0)=2 \pi F^{\prime}(t)$ для некоторого $t$, то $(0,0)=2 \pi(-\sin t, \cos t)$ - противоречие.
    
\end{task}
\begin{task}
    Показать, что если отображение $F$ строго дифференцируемо в нуле и дифференцируемо по Гато в окрестности 0 , то производная по Гато $F_{G}(x): h \mapsto F_{G}(x)[h]$ непрерывна в 0 .\\
    \textbf{Решение.} \\
    Если $F$ строго дифференцируемо в 0 , т.е.
    
    $$
    \left\|F(x)-F(y)-F^{\prime}(0)(x-y)\right\| \leq \varepsilon\|x-y\| \quad \Longrightarrow \quad\left\|F(x)-F(y)-F_{G}^{\prime}(0)(x-y)\right\| \leq \varepsilon\|x-y\|
    $$
    
    при $\|x\| \leq \delta(\varepsilon)$ и $\|y\| \leq \delta(\varepsilon)$, то константа Липшица отображения $x \mapsto F(x)-F^{\prime}(0)[x]$, обозначенная здесь через $\varepsilon$ стремится к 0 в дельта-окрестности 0 при дельта стремящемся к 0 .
    
    Поскольку отображение $\Phi: h \mapsto\left(F_{G}^{\prime}(x)-F_{G}^{\prime}(0)\right)[h]$ липшицево с константой эпсилон в окрестности 0 , то в каждой точке этой окрестности норма производной $\Phi$ не больше эпсилон. Поэтому приходим к заключению: $\left\|F_{G}^{\prime}(x)-F_{G}^{\prime}(0)\right\|$ не больше эпсилон, если $x$ в маленькой окрестности нуля. Требуемая непрерывность установлена.
\end{task}
\begin{task}
Пусть $T: L^2(0,1) \rightarrow L^2(0,1), T x(t)=\sin x(t)$. Показать, что $T$ дифференцируемо по Гато в каждой гочке, но нигде не дифференцируемо по Фреше.
\\
\textbf{Решение.} \\

\textbf{1. Дифференцируемость по Гато.} \\
Сначала зафиксируем вектор $h \in L^2(0,1)$. Нусть $\lambda \in \mathbb{R}$. Для любого $t$
$$
\lim _{\lambda \rightarrow 0} \frac{\sin [x(t)+\lambda h(t)]-\sin x(t)}{\lambda} \stackrel{(\sin x)^{\prime}=\cos x}{=} \cos x(t) \cdot h(t) .
$$

Таким же будет предел в $L^2(0,1)$, т.е.
$$
\lim _{\lambda \rightarrow 0} \int_0^1\left(\frac{\sin [x(t)+\lambda h(t)]-\sin x(t)}{\lambda}-\cos x h(t)\right)^2 d t=0,
$$

поскольку поточечная сходимость есть, а подинтегральная функция мажорируется функцией из $L^1$,
$$
\left|\frac{\sin [x(t)+\lambda h(t)]-\sin x(t)}{\lambda}\right| \leq|h(t)|.
$$

\textbf{2. Однако нет дифференцируемости по Фреше.} \\
Следуя указанию А.А. Васильевой, рассмотрим два cлучая.
\\
А) Если $\mu\{t:|\cos x(t)| \neq 0\}>0$, то существует $\varepsilon>0$ и $E=\{t:|\cos x(t)| \geq \varepsilon\}>0$. 
Пусть $E_n \subset E$, что $\mu E_n \rightarrow 0$ при $n \rightarrow \infty$. Возьмем $h_n=2 \pi \cdot 1_{E_n}$. Тогда $\sin \left(x(t)+h_n(t)\right)-\sin x(t) \equiv 0$. Поэтому
$$
\frac{\left\|\cos x(\cdot) h_n(\cdot)\right\|_{L^2}}{\left\|h_{\mathrm{n}}(\cdot)\right\|_{L^2}} \geq \frac{\varepsilon\left(\mu E_n\right)^{1 / 2}}{\left(\mu E_n\right)^{1 / 2}}=\varepsilon .
$$
В). Если $\mu\{t:|\cos x(t)| \neq 0\}=0$, то возьмем $h_n=\pi \cdot 2 \cdot 1_{E_n}$. Далее аналогично. Получается, что $T$ не дифференцируемо и по Фреше.

\end{task}
\begin{task}
Пусть $A: l_2 \rightarrow l_2$,
$$
A\left(x_1, x_2, \ldots, x_n, \ldots\right)=\left(x_1, x_2 / 2, \ldots, x_n / n, \ldots\right),
$$
$\left(y_1, \ldots, y_n, \ldots\right) \in l_2 \backslash \operatorname{Im} A$ (почему такая точка существует?). Рассмотрим задачу
$$
\sum_{n=1}^{\infty} y_n x_n \rightarrow \inf , \quad A\left(x_1, x_2, \ldots, x_n, \ldots\right)=0 .
$$

Какая точка будет точкой минимума в этой задаче? Показать, что для этой задачи принцип Лагранжа неверен. Какое из условий теоремы о необходимом условии локального минимума здесь не выполнено?
\vspace{1cm}

\textbf{Решение.} 

1) В качестве точки $y$ можно взять последовательность 
$(1,1 / 2, \ldots, 1 / n, \ldots) \in l_2$. Если $A x=y$, то $x_n=1$ для любого $n \in \mathbb{N}$, но $(1, \ldots, 1, \ldots) \notin l_2$.
\vspace{0.5cm}

2) Если $A\left(x_1, x_2, \ldots, x_n, \ldots\right)=0$, то $\frac{x_n}{n}=0$ для любого $n$. Значит, $x=0$-единственная допустимая точка, она же и будет точкой минимума.
\vspace{0.5cm}

3) Пусть $f_0(x)=\sum_{n=1}^{\infty} y_n x_n, F(x)=A(x)$. Тогда $f_0^{\prime}(x)[h]=\sum_{n=1}^{\infty} y_n h_n, F^{\prime}(x)[h]=A(h)=$ $\left(h_1, h_2 / 2, \ldots, h_n / n, \ldots\right)$. Если $z^*$ - линейный непрерывный функционал на $l_2$, то существует вектор $z=\left(z_1, \ldots, z_n, \ldots\right) \in l_2$ такой, что $z^*(x)=\sum_{n=1}^{\infty} z_n x_n$.

Таким образом, если принцип Лагранжа выполнен, то существуют $\lambda_0 \in \mathbb{R}$ и $z \in l_2$, одновременно не равные нулю, такие, что для любого $h \in l_2$ выполнено
$$
\lambda_0 \sum_{n=1}^{\infty} y_n h_n+\sum_{n=1}^{\infty} z_n \frac{h_n}{n}=0 .
$$

Значит, $\lambda_0 y_n+\frac{z_n}{n}=0, n \in \mathbb{N}$. Если $\lambda_0 \neq 0$, то $y_n=-\frac{z_n}{\lambda_0 n}$, то есть $y=A\left(-z / \lambda_0\right)$. Но $y \notin \operatorname{Im} A$ - противоречие. Если $\lambda_0=0$, то $\frac{z_n}{n}=0$ для любого $n$, поэтому $z=0$. Получили, что оба множителя Лагранжа нулевые.
\vspace{0.5cm}

4) Пространства $X=Y=l_2$ банаховы, $f_0$ и $F$ непрерывно дифференцируемы (это линейные непрерывные отображения). Но $\operatorname{Im} F^{\prime}(0)=\operatorname{Im} A$ незамкнут (он всюду плотен в $l_2$, но не совпадает с $l_2$ ).

\end{task}

\begin{task}
Задача 15. Привести пример гладких функций $f_{0}: \mathbb{R} \rightarrow \mathbb{R}, f_{1}: \mathbb{R} \rightarrow \mathbb{R}$ таких, что в задаче $f_{0}(x) 
\rightarrow \min , f_{1}(x)=0$ будет существовать точка локального минимума и в принципе Лагранжа будет $\lambda_{0}=0$ (а с $\lambda_{0} \neq 0$ принцип Лагранжа не выполнен).

Решение. Рассмотрим задачу

$$
x \rightarrow \inf , \quad x^{2}=0
$$

Единственная допустимая точка $-\hat{x}=0$. Значит, она и будет точкой минимума. Запишем функцию Лагранжа: $\mathcal{L}=\lambda_{0} x+\lambda_{1} x^{2}$. Приравнивая ее производную в $\hat{x}$ к нулю, получаем $\lambda_{0}=0$.
\end{task}
\begin{task}

    Пусть $\hat{x} \in M$ - точка локального минимума в задаче
    
    $$
    \left\{\begin{array}{l}
    f_{0}(x) \rightarrow \inf \\
    x \in M
    \end{array}\right.
    $$\\
    функция $f_{0}$ дифференцируема по Гато в точке $\hat{x}$. \\
    Верно ли, что тогда $f_{0}^{\prime}(\hat{x})[h]=0$ для любого $h \in T_{\hat{x}} M$ ?\\
    \textbf{Ответ:} нет, неверно.\\
    \textbf{Пример:}
    Пусть $M=\left\{(x, y): y=x^{2}\right\}$ (т.е. парабола на плоскости),
    $$
    f_{0}(x, y)= \begin{cases}0, & \text { если } y=x^{2} \\ x, & \text { иначе. }\end{cases}
    $$
    
    Тогда $f_0 = 0$ на M, поэтому $(0,0)$ - точка минимума $f_{0}$ на $M$. При $f_{0}(t h, t g)=t h$ при малых $t$ (если t < $\frac{g}{h}$, то $t^2h^2 < tg)$, поэтому $f_{0}^{\prime}(0,0)[(h, g)]=h$.\\ 
    Касательный вектор в $(0,0)$ к параболе $M$ имеет вид $(h, 0)$, так что на нем производная равна $h \neq 0$. 
    \end{task}
\begin{task}
Пусть $l>0$. Доказать, что допустимые экстремали в задаче

$$
\int_{0}^{1}(y \dot{x}-x \dot{y}) d t \rightarrow \max , \quad \int_{0}^{1} \sqrt{\dot{x}^{2}+\dot{y}^{2}} d t=l, \quad x(0)=x(1)=y(0)=y(1)=0, \quad \dot{x}^{2}+\dot{y}^{2}>0
$$

являются параметризацией окружности.

\textbf{Решение.} Функция Лагранжа имеет вид

$$
\int_{0}^{1}\left(\lambda_{0}(-y \dot{x}+x \dot{y})+\lambda_{1} \sqrt{\dot{x}^{2}+\dot{y}^{2}}\right) d t
$$

Значит, уравнения Эйлера имеют вид

$$
\begin{aligned}
& -\frac{d}{d t}\left(\lambda_{1} \frac{\dot{x}}{\sqrt{\dot{x}^{2}+\dot{y}^{2}}}-\lambda_{0} y\right)+\lambda_{0} \dot{y}=0, \\
& -\frac{d}{d t}\left(\lambda_{1} \frac{\dot{y}}{\sqrt{\dot{x}^{2}+\dot{y}^{2}}}+\lambda_{0} x\right)-\lambda_{0} \dot{x}=0 .
\end{aligned}
$$

Если $\lambda_{1}=0$, то $\lambda_{0} \dot{y}=0, \lambda_{0} \dot{x}=0$. Так как $\lambda_{0} \neq 0$, то $\dot{y}=0, \dot{x}=0$, что противоречит условию $\dot{x}^{2}+\dot{y}^{2}>0$.

Пусть $\lambda_{1} \neq 0$. Можно считать, что $\lambda_{1}=2$. Тогда

$$
-\frac{d}{d t} \frac{\dot{x}}{\sqrt{\dot{x}^{2}+\dot{y}^{2}}}+\lambda_{0} \dot{y}=0
$$

Пусть $\lambda_{1} \neq 0$. Можно считать, что $\lambda_{1}=2$. Тогда

$$
-\frac{d}{d t} \frac{\dot{x}}{\sqrt{\dot{x}^{2}+\dot{y}^{2}}}+\lambda_{0} \dot{y}=0
$$

$$
-\frac{d}{d t} \frac{\dot{y}}{\sqrt{\dot{x}^{2}+\dot{y}^{2}}}-\lambda_{0} \dot{x}=0
$$

откуда

$$
\frac{\dot{x}}{\sqrt{\dot{x}^{2}+\dot{y}^{2}}}=\lambda_{0} y+a, \quad \frac{\dot{y}}{\sqrt{\dot{x}^{2}+\dot{y}^{2}}}=-\lambda_{0} x+b
$$

Возведем обе части равенств в квадрат и получим

$$
1=\left(-\lambda_{0} y+a\right)^{2}+\left(\lambda_{0} x+b\right)^{2}
$$

Заметим, что $\lambda_{0} \neq 0$, иначе $\frac{d y}{d x}=$ const или $\frac{d x}{d y}=$ const, при этом $(\dot{x}, \dot{y})$ нигде не обращается в $(0,0)$. Тогда будет движение по отрезку всё время в одном направлении, что противоречит граничным условиям. А если $\lambda_{0} \neq 0$, то (3) - уравнение окружности.
\end{task}
\begin{task}
Привести пример такой задачи выпуклого программирования, что допустимая $\hat{x}-$ не есть точка минимума, но существует ненулевой набор $\left(\lambda_{0}, \ldots, \lambda_{m}\right)$, удовлетворяющий условиям а)-с) теоремы Куна-Таккера.

\begin{theorem}
    (Каруш - Кун - Таккер). Пусть $X$ --- линейное пространство, $f_0, \ldots, f_m: X \rightarrow \mathbb{R} \cup\{+\infty\}$ --- выпуклые функии.
    \begin{enumerate}
        \item (необходимое условие). Пусть $\hat{x}$ - точка минимума в задаче: 
                \\
                $\left\{\begin{array}{l}f_0(x) \rightarrow \inf \\ f_j(x) \leq 0, \quad 1 \leq j \leq m\end{array}\right.$
                \\

        Тогда существует ненулевой набор чисел $\lambda_0, \lambda_1, \ldots, \lambda_m$ со следующими свойствами: \\
            (a) $\lambda_j \geq 0,0 \leq j \leq m$ (условие неотрицательности); \\
            (b) $\lambda_j f_j(\hat{x})=0, 1 \leq j \leq m$ (условие дополняющей нежесткости); \\
            (c) $\hat{x}$ является точкой минимума функции $\mathcal{L}(x):=\sum_{j=0}^m \lambda_j f_j(x)$ (условие минимума). \\
        \item (достаточное условие). Пусть $\hat{x}-$ допустимая точка. Пусть существует набор чисел $\lambda_0, \lambda_1, \ldots, \lambda_m$ 
            со свойствами а)-с), при этом $\lambda_0>0$. Тогда $\hat{x}-$ точка минимума в рассматриваемой задаче.
        \item Пусть существует точка $\bar{x} \in X$ такая, что $f_j(\bar{x})<0,1 \leq j \leq m$ (условие Слейтера). 
        Тогда, если $\lambda_0, \lambda_1, \ldots, \lambda_m$ - ненулевой набор чисел со свойствами а)-с), то $\lambda_0>0$.
    \end{enumerate}
\end{theorem}

\textbf{Пример.}
% Если $\hat{x}$ - решение задачи на минимум $f_{0}(x)$ при условии $f_{1}(\hat{x})=0, f_{2}(\hat{x})=0$, где функционалы выпуклы, 
% то для функции Лагранжа $\mathcal{L}(x)=$ $\sum_{j \geq 0} \lambda_{j} f_{j}(x)$ справедливы условия

% a) минимум функции Лагранжа достигается на решении;

% b) $\lambda_{j} f_{j}(\hat{x})=0, j \geq 1$;

% c) $\lambda_{j} \geq 0, j \geq 0$.

Пусть $x=\left(x_{1}, x_{2}\right) \in \mathbb{R}^{2}, f_{1}(x)=-x_{1}, f_{2}(x)=x_{2}$, а 
$f_{0}(x)=x_{1}^{2}+\left(x_{2}+1\right)^{2}$. 
Тогда для функции Лагранжа ${\mathcal{L}(x)=-x_1}$ имеем: точка $\hat{x}=(0,0)$ 
- допустимая, условия а)-с) выполнены, но минимум $f_{0}(x)$ достигается в точке $(0,-1)$.


\end{task}

\begin{task}
(распределение с максимальной энтропией; см. тему про достаточное условие глобального минимума в задаче с равенствами и неравенствами). 
Пусть $\rho:[0,+\infty) \rightarrow(0,+\infty), \int_{0}^{\infty} \rho(x) d x=1$ (функция $\rho$ имеет смысл плотности распределения). Энтропией называется величина $S=-\int_{0}^{\infty} \rho(x) \ln \rho(x) d x$. Найти функцию $\rho$, для которой энтропия максимальна при заданном среднем (т.е. задано ограничение $\int_{0}^{\infty} x \rho(x) d x=C_{1}$ ).

\textbf{Решение.} Напишем задачу на экстремум:

$$
\int_{0}^{\infty} \rho(x) \ln \rho(x) d x \rightarrow \inf , \quad \int_{0}^{\infty} \rho(x) d x=1, \quad \int_{0}^{\infty} x \rho(x) d x=C_{1}
$$

Составим функцию Лагранжа с $\lambda_{0}=1$ :

$$
\begin{gathered}
	\mathcal{L}=\int_{0}^{\infty} \rho(x) \ln \rho(x) d x+\lambda_{1} \int_{0}^{\infty} \rho(x) d x+\lambda_{2} \int_{0}^{\infty} x \rho(x) d x= \\
	=\int_{0}^{\infty}\left(\rho(x) \ln \rho(x)+\lambda_{1} \rho(x)+\lambda_{2} x \rho(x)\right) d x .
\end{gathered}
$$

Найдем минимум у функции $\mathcal{L}$. Для этого при каждом фиксированном $x \in[0, \infty)$ найдем точку минимума у $f(v)=v \ln v+\lambda_{1} v+\lambda_{2} x v$. Вычислим производную: $f^{\prime}(v)=\ln v+1+\lambda_{1}+\lambda_{2} x$. Эта функция строго возрастает по $v ; f^{\prime}(v)=0$ в точке $\hat{\rho}(x)=e^{-1-\lambda_{1}-\lambda_{2} x}$. Значит, $\hat{\rho}(x)=e^{-1-\lambda_{1}-\lambda_{2} x}$ является точкой минимума $v \ln v+\lambda_{1} v+\lambda_{2} x v$.

Из условий $\int_{0}^{\infty} e^{-1-\lambda_{1}-\lambda_{2} x} d x=1$ и $\int_{0}^{\infty} x e^{-1-\lambda_{1}-\lambda_{2} x} d x=C_{1}$ находим $\lambda_{1}$ и $\lambda_{2}$ :

$$
e^{\lambda_{1}+1}=\frac{1}{\lambda_{2}}, \quad C_{1} e^{\lambda_{1}+1}=\frac{1}{\lambda_{2}^{2}}
$$

Докажем, что найденная функция будет точкой минимума в задаче. В самом деле, пусть $\rho(x)$ - допустимая функция. Тогда для любого $x \in$ $[0, \infty)$ получаем

$$
\rho(x) \ln \rho(x)+\lambda_{1} \rho(x)+\lambda_{2} x \rho(x) \geq \hat{\rho}(x) \ln \hat{\rho}(x)+\lambda_{1} \hat{\rho}(x)+\lambda_{2} x \hat{\rho}(x) .
$$

Интегрируем это неравенство и получаем $\mathcal{L}(\rho) \geq \mathcal{L}(\hat{\rho})$, т.е.

$$
\begin{aligned}
	& \int_{0}^{\infty} \rho(x) \ln \rho(x) d x+\lambda_{1} \int_{0}^{\infty} \rho(x) d x+\lambda_{2} \int_{0}^{\infty} x \rho(x) d x \geq \\
	& \geq \int_{0}^{\infty} \hat{\rho}(x) \ln \hat{\rho}(x) d x+\lambda_{1} \int_{0}^{\infty} \hat{\rho}(x) d x+\lambda_{2} \int_{0}^{\infty} x \hat{\rho}(x) d x .
\end{aligned}
$$

Воспользуемся ограничениями в задаче и получим, что $\int_{0}^{\infty} \rho(x) \ln \rho(x) d x \geq$ $\int_{0}^{\infty} \hat{\rho}(x) \ln \hat{\rho}(x) d x$.

Теперь докажем, что других точек минимума в задаче нет. Пусть $\rho_{1}$ и $\rho_{2}$ - две разные точки минимума. Обозначим минимальное значение через $m$; тогда $\int_{0}^{\infty} \rho_{1}(x) \ln \rho_{1}(x) d x=\int_{0}^{\infty} \rho_{2}(x) \ln \rho_{2}(x) d x=m$. Функции $\rho_{1}$ и $\rho_{2}$ различаются на множестве положительной меры. Заметим, что функция $\rho=\left(\rho_{1}+\rho_{2}\right) / 2$ также является допустимой. Покажем, что $\int_{0}^{\infty} \rho(x) \ln \rho(x) d x<m$.

В самом деле, функция $\varphi(v)=v \ln v$ строго выпукла, т.е. для любых $u \neq w, \lambda \in(0,1)$ выполнено $\varphi((1-\lambda) u+\lambda w)<(1-\lambda) \varphi(u)+\lambda \varphi(w)$. Это верно, так как $\varphi^{\prime \prime}(v)>0$ для любого $v>0$.

Значит, на множестве положительной меры

$$
\frac{\rho_{1}(x)+\rho_{2}(x)}{2} \ln \left(\frac{\rho_{1}(x)+\rho_{2}(x)}{2}\right)<\frac{1}{2} \rho_{1}(x) \ln \rho_{1}(x)+\frac{1}{2} \rho_{2}(x) \ln \rho_{2}(x),
$$

поэтому

$$
\int_{0}^{\infty} \rho(x) \ln \rho(x) d x<\frac{1}{2} \int_{0}^{\infty} \rho_{1}(x) \ln \rho_{1}(x) d x+\frac{1}{2} \int_{0}^{\infty} \rho_{2}(x) \ln \rho_{2}(x) d x=m .
$$
\end{task}
\begin{task}
    (аэродинамическая задача Ньютона). Найти допустимые экстремали в задаче
    $$
        \begin{cases}
            \int_0^{T_0} \frac{t}{1+u^2} d t \rightarrow \inf
            \\
            x(0)=0, x\left(T_0\right)=\xi
            \\
            \dot{x}=u
            \\
            u \geq 0
        \end{cases}
    $$

    здесь $T_0>0, \xi>0$ - заданные параметры. (Ответ для $\hat{x}(t)$ записывается в параметрическом виде: $x=x(v), t=t(v)$.)
    Доказать, что допустимая экстремаль существует и единственна, и что она будет точкой глобального минимума в данной задаче.

    \textbf{Решение.} \\
    Составим функцию Лагранжа:
    $$
        \mathcal{L}=\lambda_0 \int_0^{T_0} \frac{t}{1+u^2} d t+\int_0^{T_0} p(t)(\dot{x}-u) d t+\lambda_1 x(0)+\lambda_2\left(x\left(T_0\right)-\xi\right) .
    $$

    Необходимые условия локального минимума имеют вид $\lambda_0 \geq 0$ (условие неотрицательности), 
    $\dot{p}=0$ (уравнение Эйлера), $p(0)=\lambda_1, p\left(T_0\right)=-\lambda_2$ (условие трансверсальности),
    $$
        \min _{v \geq 0}\left(\frac{\lambda_0 t}{1+v^2}-p(t) v\right)=\frac{\lambda_0 t}{1+\hat{u}(t)^2}-p(t) \hat{u}(t)
    $$
    (принцип максимума Понтрягина).

    Из уравнения Әйлера получаем, что $p(t)=c$.
    Пусть $\lambda_0=0$. Тогда $\min _{v \geq 0}(-c v)=-c \hat{u}(t)$. Если $c=0$, то $p(t) \equiv 0$; 
    из условия трансверсальности следует, что $\lambda_1=\lambda_2=0$, то есть все множители Лагранжа нулевые. 
    Если $c>0$, то у функции $-c v$ на $[0,+\infty$ ) точки минимума нет. Если $c<0$, то $\hat{u}(t) \equiv 0$. 
    В силу граничного условия в нуле, $\hat{x}(t) \equiv 0$, что прогиворечиг с условием $x\left(T_0\right)=\xi>0$.

    Пусть $\lambda_0>0$. Без ограничения общности можно взять $\lambda_0=1$. Также обозначим $q=-c$. Получаем
    $$
        \min _{v \geq 0}\left(\frac{t}{1+v^2}+q v\right)=\frac{t}{1+\hat{u}(t)^2}+q \hat{u}(t) .
    $$

    Для фиксированного $t \in\left[0, T_0\right]$ положим $f(v)=\frac{t}{1+v^2}+q v$. 
    Если $q \leq 0$, то минимум функции $f$ не достигается, так как она строго убывает. Значит, остается случай $q>0$.
    Найдем участки монотонности функции $f$ на $\mathbb{R}_{+}$.
    Имеем: $f^{\prime}(v)=-\frac{2 t v}{\left(1+v^2\right)^2}+q$. Условие $f^{\prime}(v)=0$ эквивалентно уравнению
    \begin{equation} \label{eq201}
        q\left(1+v^2\right)^2-2 t v=0.
    \end{equation}

    В левой части стоит строго выпуклая функция, поэтому у нее количество нулей не больше 2. Также заметим, что максимальный корень строго возрастает по $t$. 
    В самом деле, если 
    ${ t_1<t_2$, $q\left(1+u_1^2\right)^2-2 t_1 u_1 = 0 }$, 
    то ${ q\left(1+u_1^2\right) ^2 - 2 t_2 u_1<0 } $. 
    Кроме того, максимальный корень стремится к $+\infty$ при $t \rightarrow+\infty$.

    Игак, либо $f$ строго возрастаег (тогда 0 гочка минимума), либо сначала возрастаег, погом убывает и затем снова возрастает. 
    Во втором случае минимум либо при $v=0$, либо в точке $u_*$, являющейся максимальным корнем уравнения \eqref{eq201}. 
    Сравним значения $f(0)$ и $f\left(u_*\right)$. Запишем неравенство $f(0) \leq f\left(u_*\right)$, получим $\frac{t u_*}{1+u_*^2} \leq q$; 
    подставим из \eqref{eq201} $t=\frac{q\left(1+u_*^2\right)^2}{2 u_*}$ и получим после вычислений $u_* \leq 1$.

    Итак, если $f$ строго возрастает или $u_*<1$, то минимум достигается в 0 ; если $u_*>1$, то минимум достигается в $u_*$. Заметим, что $u_*=1$ при $t=2 q$. 
    Так как $u_*$ строго возрастает по $t$, то при $t<2 q$ минимум функции $f$ достигается в 0 , а при $t>2 q \quad$ в $u_*$.
    Итак,
    $$
        \hat{u}(t)=\begin{cases} 
            0, \quad t<2 q, 
            \\
            u_*(t), \quad t>2 q .
        \end{cases}
    $$

    В силу условия $x(0)=0$, при $0 \leq t \leq 2 q$ получаем $\hat{x}(t)=0$.
    При $t \geq 2 q$ функцию $x(t)$ запишем параметрически. Выражая $t$ через $v$ из (2), получаем $t(v)=\frac{q}{2}\left(\frac{1}{v}+2 v+v^3\right)$. Далее, $\frac{d x}{d v}=\frac{d x}{d t} \cdot \frac{d t}{d v}=v \cdot \frac{q}{2}\left(-\frac{1}{v^2}+2+3 v^2\right)=\frac{q}{2}\left(-\frac{1}{v}+2 v+3 v^3\right)$. Значит, $x(v)=\frac{q}{2}\left(-\ln v+v^2+\frac{3}{4} v^4\right)+C$; константа $C$ находится из условия $x(1)=0$ (здесь мы вос пользовались тем, что $x$ непрерывна по $t$ и $u_*(2 q)=1$ ), т.е. $x(v)=\frac{q}{2}\left(-\ln v+v^2+\frac{3}{4} v^4-\frac{7}{4}\right)$. Игак,
    $$
        t(v)=\frac{q}{2}\left(\frac{1}{v}+2 v+v^3\right), \quad x(v)=\frac{q}{2}\left(-\ln v+v^2+\frac{3}{4} v^4-\frac{7}{4}\right) .
    $$

    Теперь покажем, что для любых $T_0>0, \xi>0$ найдется $q>0$ такое, что $x\left(T_0\right)=\xi$.
    Пусть $a=\frac{\xi}{T_0}$. Возьмем $q=2$. Покажем, что если $x_0(t)=0$ при $t \leq 2 q=4$, а при $t>4$ задано парамегрически :
    $$
        t(v)=\frac{1}{v}+2 v+v^3, \quad x_0(v)=-\ln v+v^2+\frac{3}{4} v^4-\frac{7}{4},
    $$

    то найдется такое $t_*>4$, что $x_0\left(t_*\right)=a t_*$. Затем определим $q$ из равенства $\frac{q t_*}{2}=T_0$.
    Мы уже говорили, что $\dot{x}_0(t)$ строго возрастает при $t>4$ и $\dot{x}_0(t) \underset{t \rightarrow \infty}{\rightarrow}+\infty$. 
    Значит, $x_0(t)-$ at $\underset{t \rightarrow+\infty}{\rightarrow}+\infty$. Kроме того, $x_0(4)-4 a<0$. 
    Поэтому уравнение $x_0(t)=a t$ имеет корень на $(4,+\infty)$. Единственность корня следует из строгой выпуклости $x_0$ на $[4,+\infty)$.

    Теперь покажем, что найденная экстремаль является точкой минимума. Достаточно показать, что она является точкой минимума функции $\mathcal{L}$. 
    Из условий трансверсальности следует, что $\lambda_1=-q, \lambda_2=q$. Имеем: $\mathcal{L}(x, u)=\mathcal{L}_1(x)+\mathcal{L}_2(u)$, где
    $$
    \mathcal{L}_1(x)=-\int_0^{T_0} q \dot{x} d t-q x(0)+q\left(x\left(T_0\right)-\xi\right) \equiv-q \xi T_0, 
    \quad \mathcal{L}_2(u)=\int_0^{T_0}\left(\frac{t}{1+u^2}+q u\right) d t .
    $$

    Из принципа максимума Понтрягина следует, что $\mathcal{L}_2(u) \geq \mathcal{L}_2(\hat{u})$. 
    Значит, $\mathcal{L}(x, u) \geq \mathcal{L}(\hat{x}, \hat{u})$ для любой допустимой пары $(x, u)$.

\end{task}
\begin{task} 
    (из лекций) Сделав замену $\dot{x}=u$, вывести необходимое условие сильного минимума в простейшей задаче 
    вариационного исчисления 
    (условие Вейерштрасса и непрерывность $\left.L_{\dot{x}}(t, \hat{x}(t), \dot{\hat{x}}(t))\right)$ из принципа максимума Понтрягина.

    \textbf{Решение.} 
    
    \begin{definition}
        Функция Вейерштрасса определяется по формуле: 
        \begin{equation*}
             \mathcal{E}(t, x, u, v) = L(t, x, v) - L(t, x, u) - L_{\dot x}(t, x, u) (v - u). 
        \end{equation*}
    \end{definition}

    \begin{definition}
        Пусть $\widehat x \in PC^{1}([t_0, t_1], \mathbb{R}^n)$ --- допустимая экстремаль (т.е. решение уравнения Эйлера--Лагранжа). Скажем, 
        что выполнено условие Вейерштрасса для 
        $\widehat x$, 
        если $ \mathcal{E} (t, \widehat x(t), \dot{\widehat{x}} (t), v) \geq 0$ 
        для любого 
        $t \in [t_0, t_1], v \in \mathbb{R}^n $.
    \end{definition}

    
    Задача записывается в виде
    
    $$
    \int_{t_{0}}^{t_{1}} L(t, x(t), u(t)) d t \rightarrow \inf , \quad x\left(t_{0}\right)=x_{0}, \quad x\left(t_{1}\right)=x_{1}, \quad \dot{x}=u .
    $$
    
    Функция Лагранжа имеет вид
    
    $$
    \mathcal{L}=\int_{t_{0}}^{t_{1}}\left(\lambda_{0} L(t, x(t), u(t))+p(t)(\dot{x}(t)-u(t))\right) d t+\lambda_{1} x\left(t_{0}\right)+\lambda_{2} x\left(t_{1}\right) .
    $$
    
    Условие неотрицательности: $\lambda_{0} \geq 0$.
    
    Уравнение Эйлера: $-\dot{p}(t)+\lambda_{0} L_{x}(t, \hat{x}(t), \hat{u}(t))=0$.
    
    Условие трансверсальности: $p\left(t_{0}\right)=\lambda_{1}, p\left(t_{1}\right)=-\lambda_{2}$.
    
    Принцип максимума Понтрягина: $\min _{v \in \mathbb{R}}\left(\lambda_{0} L(t, \hat{x}(t), v)-p(t) v\right)=$ $\lambda_{0} L(t, \hat{x}(t), \hat{u}(t))-p(t) \hat{u}(t)$.
    
    Так как $L$ гладкая и минимум берется по $v \in \mathbb{R}$, то получаем $\lambda_{0} L_{\dot{x}}(t, \hat{x}(t), \hat{u}(t))-$ $p(t)=0$.
    
    Если $\lambda_{0}=0$, то отсюда $p=0$; в силу условий трансверсальности, $\lambda_{1}=\lambda_{2}=0$, то есть все множители Лагранжа нулевые.
    
    Итак, $\lambda_{0}>0$. Можно считать, что $\lambda_{0}=1$. Так как $\dot{\hat{x}}=\hat{u}$, то $L_{\dot{x}}(t, \hat{x}(t), \dot{\hat{x}}(t))=$ $p(t)$. В теореме о необходимом условии сильного минимума в задаче оптимального управления функция $p$ кусочно непрерывно-дифференцируемая и, значит, непрерывная. Отсюда получаем непрерывность $t \mapsto L_{\dot{x}}(t, \hat{x}(t), \dot{\hat{x}}(t))$.
    
    Еще раз запишем принцип максимума Понтрягина: для любого $v \in \mathbb{R}$
    
    $$
    L(t, \hat{x}(t), v)-L_{\dot{x}}(t, \hat{x}(t), \dot{\hat{x}}(t)) v \geq L(t, \hat{x}(t), \hat{u}(t))-L_{\dot{x}}(t, \hat{x}(t), \dot{\hat{x}}(t)) \hat{u}(t)
    $$
    
    Подставим $\hat{u}=\dot{\hat{x}}$, перенесем все в левую часть и получим условие Вейерштрасса.
    
    \end{task}
\begin{task}
    Показать, что если $L$ явно не зависит от $x$ (т.е. $L=L(t, \dot{x}(t)))$, то условие Вейерштрасса будет достаточным условием глобального минимума.

    \textbf{Решение.} В силу уравнения Эйлера, $L_{\dot{x}}(t, \dot{\hat{x}}(t)) \equiv c$.
    
    Пусть $x$ - произвольная допустимая функция. В силу условия Вейерштрасса
    
    $$
    \int_{t_{0}}^{t_{1}} \mathcal{E}(t, \hat{x}(t), \dot{\hat{x}}(t), \dot{x}(t)) d t \geq 0
    $$
    
    откуда
    
    $$
    \int_{t_{0}}^{t_{1}}\left(L(t, \dot{x}(t))-L(t, \dot{\hat{x}}(t))-L_{\dot{x}}(t, \dot{\hat{x}}(t))(\dot{x}(t)-\dot{\hat{x}}(t))\right) d t \geq 0
    $$
    
    Значит,
    
    $$
    \int_{t_{0}}^{t_{1}} L(t, \dot{x}(t)) d t \geq \int_{t_{0}}^{t_{1}} L(t, \dot{\hat{x}}(t)) d t+\int_{t_{0}}^{t_{1}} c(\dot{x}(t)-\dot{\hat{x}}(t)) d t=
    $$
    
    $$
    =\int_{t_{0}}^{t_{1}} L(t, \dot{\hat{x}}(t)) d t+\left.c x\right|_{t_{0}} ^{t_{1}}-\left.c \hat{x}\right|_{t_{0}} ^{t_{1}}=\int_{t_{0}}^{t_{1}} L(t, \dot{\hat{x}}(t)) d t
    $$
    
    так как $x\left(t_{0}\right)=\hat{x}\left(t_{0}\right)$ и $x\left(t_{1}\right)=\hat{x}\left(t_{1}\right)$.
    
\end{task}
\begin{task}
    Рассмотрим задачу $\int_0^\pi\left(\dot{x}^2-x^2-x^4\right) d t \rightarrow$ inf, $x(0)=x(\pi)=$ 0. Показать, что для $\hat{x}=0$ выполнено yсиленное условие Лежантра, условие Якоби, при этом $\hat{x}=0$ не является точкой слабого минимума.
    
    \textbf{Peшение.} Имеем $\hat{L}_{\dot{x} \dot{x}}(t)=2, \hat{L}_{\dot{x} x}=0, \hat{L}_{x x}=-2$. 3начит, выполнено усиленное условие Лежандра. Уравнение Якоби имеет вид $\ddot{h}+h=0$; его нетривиальное решение, зануляющееся при $t=0$, имеет вид $h(t)=a \sin t$, $a \neq 0$. Torда $h(t) \neq 0$ при $t \in(0, \pi)$, но $h(\pi)=0$. Значит, выполнено условие Якоби, но не усиленное.
    Boзьмем $x(t)=\varepsilon \sin t$. Toгдa
    $$
    \begin{gathered}
    \int_0^\pi\left(\dot{x}^2-x^2-x^4\right) d t=\varepsilon^2 \int_0^\pi\left(\cos ^2 t-\sin ^2 t\right) d t-\varepsilon^4 \int_0^\pi \sin ^4 t d t= \\
    =\varepsilon^2 \int_0^\pi \cos 2 t d t-\varepsilon^4 \int_0^\pi \sin ^4 t d t=-\varepsilon^4 \int_0^\pi \sin ^4 t d t<0 .
    \end{gathered}
    $$
    
    Так как $\varepsilon>0$ может быть сколь угодно мало, то слабого минимума нет.
\end{task}
\begin{task}
    $$F(x)=\int_{0}^{3 / 2}\left(\dot{x}^{3}+2 x\right) d t \rightarrow \inf , x(0)=0, x(3 / 2)=1$$. 
    
    Доказать, что выполнено (неусиленное) условие Лежандра, усиленное условие Якоби, а допустимая экстремаль не дает слабый минимум.
    
    Примечание. Усиленное условие Якоби предполагает 2й порядок этого линейного уравнения. В моем файле TU усиленное условие Якоби трактовалось при выполнении усиленного условия Лежандра, которое обеспечивает 2й порядок уравнения Якоби.\\
    \textbf{Решение.} \\
    Уравнение экстремалей $f_{x}=\frac{d}{d t} f_{\dot{x}}$ для $f=\dot{x}^{3}+2 x$ имеет вид $2=6 \dot{x} \ddot{x}$, т.е. $3 y \dot{y}=1$ для $y=\dot{x}$. Имеем $3 y d y=d t$, т.е. $\frac{3}{2} y^{2}=t+C$. При $C=0$ имеем $\dot{x}(t)=\left(\frac{2}{3} t\right)^{1 / 2}$, что при условии $x(0)=0$ дает
    
    $$
    \hat{x}(t)=\left(\frac{2}{3}\right)^{1 / 2} t^{3 / 2} \cdot \frac{2}{3}=\left(\frac{2}{3} t\right)^{3 / 2} \Longrightarrow \hat{x}\left(\frac{3}{2}\right)=1
    $$
    
    т.е. допустимую экстремаль $\hat{x}$. Имеем
    
    $$
    \begin{aligned}
    & F(\hat{x}+h)-F(\hat{x})=\int_{0}^{3 / 2}\left\{\left([\dot{\hat{x}}+\dot{h}]^{3}+2[\hat{x}+h]\right)-\left([\dot{\hat{x}}]^{3}+2[\hat{x}]\right)\right\} d t= \\
    & \int_{0}^{3 / 2}\left\{3(\dot{\hat{x}})^{2} \dot{h}+3 \dot{\hat{x}}(\dot{h})^{2}+(\dot{h})^{3}+2 h\right\} d t=\int_{0}^{3 / 2}\left\{3 \dot{\hat{x}}(\dot{h})^{2}+(\dot{h})^{3}\right\} d t
    \end{aligned}
    $$
    
    Последнее равенство верно, т.к. на экстремали $\hat{x}$ линейная по $h$ часть разности $F(\hat{x}+h)-F(\hat{x})$ равна нулю. Таким образом,
    
    $$
    F(\hat{x}+h)-F(\hat{x})=\int_{0}^{3 / 2}\left\{3 \dot{\hat{x}}(\dot{h})^{2}+(\dot{h})^{3}\right\} d t=\int_{0}^{3 / 2}\left\{3\left(\frac{2 t}{3}\right)^{1 / 2}(\dot{h})^{2}+(\dot{h})^{3}\right\} d t
    $$
    
    Заметим, что усиленное условие Лежандра $\hat{f}_{\dot{x} \dot{x}}(t)>0$ при $t \in[0,3 / 2]$ нарушается при $t=0$. Учитывая это есть резон быстро уйти от нуля, взяв функцию $h$ кусочно-линейной, такую, что $h(0)=h(3 / 2)=0$, а $\dot{h}=-\varepsilon<0$ при $0<t<\delta \ll 1$ и $\dot{h}=a>0$ при $0<t<\delta$. Тогда $h(t)=a(t-3 / 2)$ при $t>\delta$ и $a \approx \frac{2}{3} \delta$. Отсюда получаем
    
    $$
    \begin{gathered}
    F(\hat{x}+h)-F(\hat{x}) \approx \int_{0}^{\delta} \sqrt{6} t^{1 / 2} \varepsilon^{2} d t-\varepsilon^{3} \delta+\int_{0}^{3 / 2} \sqrt{6} t^{1 / 2} a^{2} d t+\frac{3}{2} a^{3}= \\
    \varepsilon^{3} \delta\left\{\frac{2}{3} \sqrt{6} \frac{\delta^{1 / 2}}{\varepsilon}+\left(\frac{3}{2}\right)^{3 / 2}(2 / 3)^{2} \frac{\delta}{\varepsilon}+\frac{3}{2}(2 / 3)^{3} \delta^{2}\right\}-\varepsilon^{3} \delta<0 \quad \text { при } \quad \frac{\delta^{1 / 2}}{\varepsilon} \ll 1 .
    \end{gathered}
    $$
\end{task}
\begin{task}

$F(x)=\int_0^1\left(\dot{x}^2-x \dot{x}^3\right) d t \rightarrow extr , x(0)=x(1)=0$. Показать, что для экстремали $\hat x = 0$ выполнено усиленное условие Лежандра, 
усиленное условие Якоби, условие Вейерштрасса (не усиленное), и $\hat x$ не является точкой сильного минимума. 

\textbf{Решение.} \\   
Имеем $\hat{L}_{\dot{x} \dot{x}}(t)=2, \hat{L}_{\dot{x} x}(t)=0, \hat{L}_{x x}(t)=0$. Значит, выполнено усиленное условие Лежандра. Уравнение Якоби имеет вид $\ddot{h}=0$. Если $h$ - нетривиальное решение и $h(0)=0$, то $h(t)=a t$, где $a \neq 0$. Эта функция зануляется только при $t=0$. Значит, выполнено усиленное условие Якоби.

Далее, $\mathcal{E}(t, \hat{x}(t), \dot{\hat{x}}(t), v)=v^{2} \geq 0$. Значит, выполнено условие Вейерштрасса.

Теперь покажем, что $\hat{x}$ не является точкой сильного минимума. Пусть $R>0,0<\delta<\frac{1}{2}$. Положим

$$
\dot{h}(t)=\left\{\begin{array}{l}
R, \quad 0 \leq t<\delta \\
-\frac{R \delta}{1-\delta}, \quad \delta<t \leq 1
\end{array}\right.
$$

$h(t)=\int_{0}^{t} \dot{h}(s) d s$. Тогда

$$
h(t)=\left\{\begin{array}{l}
R t, \quad 0 \leq t \leq \delta \\
\frac{R \delta}{1-\delta}(1-t), \quad \delta \leq t \leq 1
\end{array}\right.
$$

Получаем

$$
\begin{aligned}
& \int_{0}^{1}\left(\dot{h}^{2}-h \dot{h}^{3}\right) d t=\int_{0}^{\delta}\left(R^{2}-R^{4} t\right) d t+ \\
& +\int_{\delta}^{1}\left(\frac{R^{2} \delta^{2}}{(1-\delta)^{2}}-\frac{R^{4} \delta^{4}}{(1-\delta)^{4}}(1-t)\right) d t \leq \\
& \leq R^{2} \delta-\frac{R^{4} \delta^{2}}{2}+C_{1} R^{2} \delta^{2}+C_{2} R^{4} \delta^{4},
\end{aligned}
$$

где $C_{1}, C_{2}$ - положительные константы. Сделав замену $R \delta=\varepsilon$, получим

$$
\int_{0}^{1}\left(\dot{h}^{2}-h \dot{h}^{3}\right) d t \leq R \varepsilon-\frac{R^{2} \varepsilon^{2}}{2}+C_{1} \varepsilon^{2}+C_{2} \varepsilon^{4}
$$

Если $R=\frac{1}{\varepsilon^{2}}$, то при малых $\varepsilon$ получим отрицательное число. При этом $\|h\|_{C}=\varepsilon$; это число можно выбрать сколь угодно малым, так что сильного минимума нет.


\end{task}
\begin{task}
Доказать, используя поле экстремалей, что допустимая экстремаль в задаче

$$
\int_{t_{0}}^{t_{1}} \frac{\sqrt{1+\dot{x}^{2}}}{x} d t \rightarrow \operatorname{extr}, \quad x\left(t_{0}\right)=x_{0}, \quad x\left(t_{1}\right)=x_{1}, \quad x>0
$$

является точкой глобального минимума (здесь $x_{0}>0, x_{1}>0$ ).

\textbf{Решение.} Напомним, что геодезические на плоскости Лобачевского - это дуги окружностей с центром на оси $t$.

Пусть $\hat{x}(t)=\sqrt{c^{2}-(t-a)^{2}}, t_{*}<t_{0}, c^{2}-\left(t_{*}-a\right)^{2}>0, x_{*}=\sqrt{c^{2}-\left(t_{*}-a\right)^{2}}$. Определим семейство экстремалей (решений уравнения Эйлера) $x(t, \alpha)$ таких, что $x\left(t_{*}, \alpha\right)=x_{*}, \dot{x}\left(t_{*}, \alpha\right)=$ $\alpha, \alpha \in \mathbb{R}$.

Сначала заметим, что для любого $\alpha \in \mathbb{R}$ геодезическая $x\left(t_{*}, \alpha\right)$ существует, при этом абсцисса центра окружности и ее радиус гладко зависят от $\alpha$. В самом деле, центр окружности получается следующим образом: проводим к прямой $x=x_{*}+\alpha\left(t-t_{*}\right)$ перпендикуляр и находим его точку пересечения с осью $t$.

Если $\tau>t_{*}, \xi>0$, то существует единственная геодезическая, проходящая через $\left(t_{*}, x_{*}\right)$ и $(\tau, \xi)$, при этом абсцисса центра окружности и радиус гладко зависит от $(\tau, \xi)$ (а значит, и $\alpha)$. Действительно, центр окружности получается следующим образом: проводим к отрезку с концами в $\left(t_{*}, x_{*}\right)$ и $(\tau, \xi)$ серединный перпендикуляр и находим его точку пересечения с осью $t$.

Осталось заметить, что $L_{\dot{x} \dot{x}}>0$, поэтому $L$ выпукла по $\dot{x}$. Значит, по достаточному условию, экстремаль будет точкой глобального минимума.
\end{task}
\begin{task}
Доказать, используя поле экстремалей, что допустимая экстремаль в задаче

$$
\int_{t_{0}}^{t_{1}} \frac{\sqrt{1+\dot{x}^{2}}}{\sqrt{x}} d t \rightarrow \operatorname{extr}, \quad x\left(t_{0}\right)=x_{0}, \quad x\left(t_{1}\right)=x_{1}, \quad x>0
$$

является точкой глобального минимума (здесь $x_{0}>0, x_{1}>0$ ).

\textbf{Peшение.} Мы уже вычисляли экстремали в параметрическом виде:

$$
x=c(1-\cos \tau), \quad t-a=c(\tau-\sin \tau)
$$

где $a \in \mathbb{R}, c>0, \tau \in[0,2 \pi]$. Будем их обозначать $x(t, a, c)$.

Наша цель доказать: допустимая экстремаль будет точкой глобального минимума.

Сначала покажем, что $\dot{x}(t, a, c)$ строго убывает, при этом принимает все вещественные значения. В самом деле, $t(\tau)$ строго возрастает; $\frac{d x}{d t}=\frac{d x}{d \tau} \cdot \frac{d \tau}{d t}=\frac{\sin \tau}{1-\cos \tau} ;$ при $\tau \rightarrow+0$ предел равен $+\infty$, при $\tau \rightarrow 2 \pi-0$ предел $-\infty$. Производная по $\tau$ от $\frac{\sin \tau}{1-\cos \tau}$ равна $-\frac{1}{1-\cos \tau}<0$, так что $\ddot{x}<0$.

Утверждение. Для любых $t_{*}<t_{* *}, x_{*}>0, x_{* *}>0$ существуют единственные $c>0, a \in \mathbb{R}$ такие, что $x\left(t_{*}, a, c\right)=x_{*}, x\left(t_{* *}, a, c\right)=x_{* *}$, при этом $a$ и $c$ гладко зависят от $t_{*}, x_{*}, t_{* *}, x_{* *}$.

Доказательство. Пусть без ограничения общности $x_{*} \leq x_{* *}$.

Пусть $k=\frac{x_{* *}-x_{*}}{t_{* *}-t_{*}}, l$ - длина отрезка $\left[\left(t_{*}, x_{*}\right),\left(t_{* *}, x_{* *}\right)\right], \gamma=l / x_{*}$. Возьмем экстремаль $x(\cdot, 0,1)$ и покажем, что существуют такие $t^{\prime}<t^{\prime \prime}$, что $\frac{x\left(t^{\prime \prime}, 0,1\right)-x\left(t^{\prime}, 0,1\right)}{t^{\prime \prime}-t^{\prime}}=k$ и отношение длины отрезка $\left[\left(t^{\prime}, x\left(t^{\prime}, 0,1\right)\right),\left(t^{\prime \prime}, x\left(t^{\prime \prime}, 0,1\right)\right)\right]$ к $x\left(t^{\prime}, 0,1\right)$ равно $\gamma$; при этом $t^{\prime}$ и $t^{\prime \prime}$ определяются однозначно и гладко зависят от $k$ и $\gamma$.

\begin{figure}[h!]
    \centering
    \includegraphics[width=0.99\linewidth]{tasks/task27/pic1.jpg}
\end{figure}

(Затем полагаем $c=x_{*} / x\left(t^{\prime}, 0,1\right), a=t_{*}-c t^{\prime}$ и получаем утверждение.)

Так как $\dot{x}(t, 0,1)$ строго убывает по $t$ и принимает все вещественные значения, то существует единственное $\tilde{t}$ такое, что $z(t)=x(\tilde{t}, 0,1)+k(t-\tilde{t})$ - касательная. Далее рассматриваем уравнение $x(t, 0,1)=z(t)-r$, где $r>0$. Так как вторая производная $x$ строго отрицательна, то количество корней не более 2 . Если решений ровно 2 , то получаем хорду $\left[\left(t_{r}^{\prime}, x_{r}^{\prime}\right),\left(t_{r}^{\prime \prime}, x_{r}^{\prime \prime}\right)\right]$ (ее концы - точки пересечения циклоиды и графика $z(t)-r$ ). При этом $t_{r}^{\prime}$ гладко зависит от $r$ (по теореме о неявной функции) и строго убывает по $r$ (а значит, и $x_{r}^{\prime}$ ), длина хорды $l_{r}$ строго возрастает по $r$; $x_{r}^{\prime}$ меняется от $x(\tilde{t}, 0,1)$ до 0 , а $l_{r}$ - от 0 до некоторого положительного числа. Значит, $l_{r} / x_{r}^{\prime}$ строго возрастает от 0 до $+\infty$ и при некотором $r>0$ равно $\gamma$ (это число $r$ единственно и гладко зависит от $\gamma$ ). Тем самым найдены $t^{\prime}$ и $t^{\prime \prime} . \diamond$

Докажем, что экстремаль в задаче о брахистохроне (с положительными $x_{0}, x_{1}$ ) будет точкой глобального минимума, используя поле экстремалей.

Пусть $t_{*}<t_{0}$ (достаточно близко к $t_{0}$, так, чтобы экстремаль гладко продолжалась в эту точку). Пусть $\hat{x}\left(t_{*}\right)=x_{*}>0$. Рассмотрим множество всех циклоид $x(t, b)$, проходящих через $\left(t_{*}, x_{*}\right)$; в качестве параметра $b$ можно взять $\dot{x}\left(t_{*}, b\right)=b$.

Пусть $\delta>0$ - достаточно малое число, $t_{*}<\tau<t_{1}+\delta, \xi>0$. В силу доказанного утверждения, существует (единственная) циклоида, проходящая через ( $t_{*}, x_{*}$ ) и $(\tau, \xi)$, при этом параметры, ее определяющие, гладко зависят от $\tau$, $\xi$. Значит, $b(\tau, \xi)$ - гладкая функция. Поэтому построенное семейство экстремалей образует центральное поле.

Наконец, $L_{\dot{x} \dot{x}}>0$, т.е. $L$ выпукла по $\dot{x}$. Поэтому по достаточному условию экстремаль будет точкой глобального минимума.
\end{task}
\begin{task}
    Рассмотрим задачу

    $$
    \int_{-T_{0}}^{T_{0}} x \sqrt{\dot{x}^{2}+1} d t \rightarrow \min , x\left(T_{0}\right)=x\left(-T_{0}\right)=\xi .
    $$
    
    1) Выписать уравнение Якоби, подобрать одно из его решений, затем найти общее решение.
    
    2) Пусть допустимых экстремалей две. Доказать, что одна из них является точкой сильного минимума, а вторая не является точкой слабого минимума.


    \textbf{Peшение.} 
    У этой задачи мы уже нашли экстремали: $\hat{x}(t)=\frac{1}{c} \mathrm{ch} c t$, где $c>0$ находится из условия с $c T_{0}=c \xi$. Напомним, что существует $\xi_{*}>0$ такое, что при $\xi<\xi_{*}$ решений нет, при $\xi=\xi_{*}$ есть ровно одно решение, при $\xi>\xi_{*}$ есть ровно два решения. При этом в случае $\xi=\xi_{*}$ число $c=c_{*}$ удовлетворяет равенству
    
    $$
    \operatorname{ch} c_{*} T_{0}-c_{*} T_{0} \cdot \operatorname{sh} c_{*} T_{0}=0
    $$
    
    В случае $\xi>\xi_{*}$ выполнено $\operatorname{ch} c_{*} T_{0}<\xi T_{0}$, поэтому одно из решений уравнения $\operatorname{ch} c T_{0}=c \xi$ будет больше $c_{*}$, а второе меньше.
    
    Выпишем уравнение Якоби.
    
    Имеем: $\dot{\hat{x}}(t)=\operatorname{sh} c t, \sqrt{1+\dot{\hat{x}}^{2}(t)}=\operatorname{ch} c t$. Отсюда
    
    $$
    \hat{L}_{\dot{x} \dot{x}}(t)=\frac{1}{c} \cdot \frac{1}{\operatorname{ch}^{2} c t}, \quad \hat{L}_{x \dot{x}}(t)=\frac{\operatorname{sh} c t}{\operatorname{ch} c t}, \quad \hat{L}_{x x}(t)=0 .
    $$
    
    Значит, уравнение Якоби имеет вид
    
    $$
    -\frac{d}{d t}\left(\frac{1}{c} \cdot \frac{1}{\operatorname{ch}^{2} c t} \dot{h}+\frac{\operatorname{sh} c t}{\operatorname{ch} c t} h\right)+\frac{\operatorname{sh} c t}{\operatorname{ch} c t} \dot{h}=0
    $$
    
    После преобразований получаем
    
    $$
    \ddot{h}-2 c \frac{\operatorname{sh} c t}{\operatorname{ch} c t} \dot{h}+c^{2} h=0 \text {. }
    $$
    
    Одно из решений угадывается: $h_{0}(t)=\mathrm{sh} c t$.
    
    Общее решение найдем через определитель Вронского. Напомним, что если $h, w$ - два решения уравнения, то определитель Вронского - это
    
    $$
    W=W(h, w)=\left|\begin{array}{cc}
    h & w \\
    \dot{h} & \dot{w}
    \end{array}\right|=h \dot{w}-w \dot{h} .
    $$
    
    Тогда $\dot{W}=h \ddot{w}-w \ddot{h}=2 c \frac{\mathrm{sh} c t}{\mathrm{ch} c t} W$ (мы воспользовались тем, что $h$ и $w-$ решения дифференциального уравнения). Решим это дифференциальное уравнение на $W$ :
    
    $$
    \dot{W} \cdot \operatorname{ch} c t-2(\operatorname{ch} c t)^{\prime} W=0
    $$
    поделив на $\mathrm{ch}^{3} c t$, получим ( $\left.W \mathrm{ch}^{-2} c t\right)^{\prime}=0$, откуда $W=a \mathrm{ch}^{2} c t$, где $a \in \mathbb{R}$.
    
    Теперь найдем $h$ через $W\left(h_{0}, h\right)$ :
    
    $$
    \dot{h} \operatorname{sh} c t-h(\operatorname{sh} c t)^{\prime}=a \operatorname{ch}^{2} c t
    $$
    
    или
    
    $$
    (h / \operatorname{sh} c t)^{\prime}=a \frac{\operatorname{ch}^{2} c t}{\operatorname{sh}^{2} c t}
    $$
    
    Если $h=z \operatorname{sh} c t$, то $\dot{z}=a \frac{\mathrm{ch}^{2} c t}{\mathrm{sh}^{2} c t}$. Заметим, что $(\operatorname{ch} s / \operatorname{sh} s)^{\prime}=1-(\operatorname{ch} s / \operatorname{sh} s)^{2}$. Отсюда $z=$ $\frac{a}{c}\left(c t-\frac{\mathrm{ch} c t}{\mathrm{sh} c t}\right)+b$, или $h=\frac{a}{c}(c t \mathrm{sh} c t-\mathrm{ch} c t)+b \operatorname{sh} c t$.
    
    Итак, общее решение уравнения Якоби является линейной комбинацией
    
    $$
    h_{0}(t)=\operatorname{sh} c t, \quad h_{1}=\operatorname{ch} c t-c t \operatorname{sh} c t \text {. }
    $$
    
    Функция $h_{0}$ нечетная и строго возрастающая, функция $h_{1}$ четная, $h_{1}(0)>0, h_{1}$ строго возрастает при $t \leq 0$ и строго убывает при $t \geq 0$.
    
    Напомним, что для одного из решений уравнения $\operatorname{ch} c T_{0}=c \xi$ выполнено $c_{1}<c_{*}$, а для другого выполнено $c_{2}>c_{*}$. Значит, при $c=c_{1}$ функция $h_{1}$ будет строго положительной на $\left[-T_{0}, T_{0}\right]$, а при $c=c_{2}$ выполнено $h_{1}\left(-T_{0}\right)<0$.
    
    Пусть $c=c_{1}$. Докажем, что любая нетривиальная линейная комбинация $h_{0}$ и $h_{1}$ имеет не более одного нуля на $\left[-T_{0}, T_{0}\right]$. Для $\alpha h_{0}$ и $\alpha h_{1}$ это понятно. Учитывая четность и нечетность, получаем: достаточно доказать, что при $\alpha>0$ уравнение $h_{1}=\alpha h_{0}$ имеет не более одного корня на $\left[-T_{0}, T_{0}\right]$. В самом деле, $h_{1}(t)>0, h_{0}(t) \leq 0$ при $t \in\left[-T_{0}, 0\right]$. На отрезке $\left[0, T_{0}\right]$ функция $h_{0}$ строго возрастает, а $h_{1}$ строго убывает. Значит, количество корней не более одного.
    
    \begin{figure}[h!]
        \centering
        \includegraphics[width=0.99\linewidth]{tasks/task28/pic2.jpg}
    \end{figure}
    
    Таким образом, при $c=c_{1}$ выполнено усиленное условие Якоби.
    
    Пусть $c=c_{2}$. Покажем, что условие Якоби не выполнено. Пусть $h=h_{1}-\alpha h_{0}, h\left(-T_{0}\right)=0$. Тогда $\alpha>0$. Значит, $h(0)>0, h\left(T_{0}\right)<0$. По теореме о промежуточных значениях, $h(\tau)=0$ для некоторого $\tau \in\left(0, T_{0}\right)$.
    
    \begin{figure}[h!]
        \centering
        \includegraphics[width=0.99\linewidth]{tasks/task28/pic3.jpg}
    \end{figure}
    
        
    
    Осталось заметить, что при $x>0$ будет выполнено $L_{\dot{x} \dot{x}}=\frac{x}{\left(1+\dot{x}^{2}\right)^{3 / 2}}>0$. Значит, выполнено усиленное условие Лежандра и усиленное условие Вейерштрасса. Поэтому в первом случае будет сильный минимум, а во втором не будет слабого минимума.

\end{task}

\end{document}
