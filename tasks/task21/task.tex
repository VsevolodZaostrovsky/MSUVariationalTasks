\begin{task} 
    (из лекций) Сделав замену $\dot{x}=u$, вывести необходимое условие сильного минимума в простейшей задаче 
    вариационного исчисления 
    (условие Вейерштрасса и непрерывность $\left.L_{\dot{x}}(t, \hat{x}(t), \dot{\hat{x}}(t))\right)$ из принципа максимума Понтрягина.

    \textbf{Решение.} 
    
    \begin{definition}
        Функция Вейерштрасса определяется по формуле: 
        \begin{equation*}
             \mathcal{E}(t, x, u, v) = L(t, x, v) - L(t, x, u) - L_{\dot x}(t, x, u) (v - u). 
        \end{equation*}
    \end{definition}

    \begin{definition}
        Пусть $\widehat x \in PC^{1}([t_0, t_1], \mathbb{R}^n)$ --- допустимая экстремаль (т.е. решение уравнения Эйлера--Лагранжа). Скажем, 
        что выполнено условие Вейерштрасса для 
        $\widehat x$, 
        если $ \mathcal{E} (t, \widehat x(t), \dot{\widehat{x}} (t), v) \geq 0$ 
        для любого 
        $t \in [t_0, t_1], v \in \mathbb{R}^n $.
    \end{definition}

    
    Задача записывается в виде
    
    $$
    \int_{t_{0}}^{t_{1}} L(t, x(t), u(t)) d t \rightarrow \inf , \quad x\left(t_{0}\right)=x_{0}, \quad x\left(t_{1}\right)=x_{1}, \quad \dot{x}=u .
    $$
    
    Функция Лагранжа имеет вид
    
    $$
    \mathcal{L}=\int_{t_{0}}^{t_{1}}\left(\lambda_{0} L(t, x(t), u(t))+p(t)(\dot{x}(t)-u(t))\right) d t+\lambda_{1} x\left(t_{0}\right)+\lambda_{2} x\left(t_{1}\right) .
    $$
    
    Условие неотрицательности: $\lambda_{0} \geq 0$.
    
    Уравнение Эйлера: $-\dot{p}(t)+\lambda_{0} L_{x}(t, \hat{x}(t), \hat{u}(t))=0$.
    
    Условие трансверсальности: $p\left(t_{0}\right)=\lambda_{1}, p\left(t_{1}\right)=-\lambda_{2}$.
    
    Принцип максимума Понтрягина: $\min _{v \in \mathbb{R}}\left(\lambda_{0} L(t, \hat{x}(t), v)-p(t) v\right)=$ $\lambda_{0} L(t, \hat{x}(t), \hat{u}(t))-p(t) \hat{u}(t)$.
    
    Так как $L$ гладкая и минимум берется по $v \in \mathbb{R}$, то получаем $\lambda_{0} L_{\dot{x}}(t, \hat{x}(t), \hat{u}(t))-$ $p(t)=0$.
    
    Если $\lambda_{0}=0$, то отсюда $p=0$; в силу условий трансверсальности, $\lambda_{1}=\lambda_{2}=0$, то есть все множители Лагранжа нулевые.
    
    Итак, $\lambda_{0}>0$. Можно считать, что $\lambda_{0}=1$. Так как $\dot{\hat{x}}=\hat{u}$, то $L_{\dot{x}}(t, \hat{x}(t), \dot{\hat{x}}(t))=$ $p(t)$. В теореме о необходимом условии сильного минимума в задаче оптимального управления функция $p$ кусочно непрерывно-дифференцируемая и, значит, непрерывная. Отсюда получаем непрерывность $t \mapsto L_{\dot{x}}(t, \hat{x}(t), \dot{\hat{x}}(t))$.
    
    Еще раз запишем принцип максимума Понтрягина: для любого $v \in \mathbb{R}$
    
    $$
    L(t, \hat{x}(t), v)-L_{\dot{x}}(t, \hat{x}(t), \dot{\hat{x}}(t)) v \geq L(t, \hat{x}(t), \hat{u}(t))-L_{\dot{x}}(t, \hat{x}(t), \dot{\hat{x}}(t)) \hat{u}(t)
    $$
    
    Подставим $\hat{u}=\dot{\hat{x}}$, перенесем все в левую часть и получим условие Вейерштрасса.
    
    \end{task}