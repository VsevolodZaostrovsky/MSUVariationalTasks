\begin{task}
Доказать, используя поле экстремалей, что допустимая экстремаль в задаче

$$
\int_{t_{0}}^{t_{1}} \frac{\sqrt{1+\dot{x}^{2}}}{x} d t \rightarrow \operatorname{extr}, \quad x\left(t_{0}\right)=x_{0}, \quad x\left(t_{1}\right)=x_{1}, \quad x>0
$$

является точкой глобального минимума (здесь $x_{0}>0, x_{1}>0$ ).

\textbf{Решение.} Напомним, что геодезические на плоскости Лобачевского - это дуги окружностей с центром на оси $t$.

Пусть $\hat{x}(t)=\sqrt{c^{2}-(t-a)^{2}}, t_{*}<t_{0}, c^{2}-\left(t_{*}-a\right)^{2}>0, x_{*}=\sqrt{c^{2}-\left(t_{*}-a\right)^{2}}$. Определим семейство экстремалей (решений уравнения Эйлера) $x(t, \alpha)$ таких, что $x\left(t_{*}, \alpha\right)=x_{*}, \dot{x}\left(t_{*}, \alpha\right)=$ $\alpha, \alpha \in \mathbb{R}$.

Сначала заметим, что для любого $\alpha \in \mathbb{R}$ геодезическая $x\left(t_{*}, \alpha\right)$ существует, при этом абсцисса центра окружности и ее радиус гладко зависят от $\alpha$. В самом деле, центр окружности получается следующим образом: проводим к прямой $x=x_{*}+\alpha\left(t-t_{*}\right)$ перпендикуляр и находим его точку пересечения с осью $t$.

Если $\tau>t_{*}, \xi>0$, то существует единственная геодезическая, проходящая через $\left(t_{*}, x_{*}\right)$ и $(\tau, \xi)$, при этом абсцисса центра окружности и радиус гладко зависит от $(\tau, \xi)$ (а значит, и $\alpha)$. Действительно, центр окружности получается следующим образом: проводим к отрезку с концами в $\left(t_{*}, x_{*}\right)$ и $(\tau, \xi)$ серединный перпендикуляр и находим его точку пересечения с осью $t$.

Осталось заметить, что $L_{\dot{x} \dot{x}}>0$, поэтому $L$ выпукла по $\dot{x}$. Значит, по достаточному условию, экстремаль будет точкой глобального минимума.
\end{task}