\begin{task} 
    Пусть $M=\left\{\left(x_{1}, x_{2}\right) \in \mathbb{R}^{2}: x_{1}>0, x_{2}=x_{1}^{2}\right\}, f: \mathbb{R}^{2} \rightarrow \mathbb{R}$,

    $$
    f\left(x_{1}, x_{2}\right)= \begin{cases}1, & \left(x_{1}, x_{2}\right) \in M \\ 0, & \left(x_{1}, x_{2}\right) \notin M\end{cases}
    $$
    
    Показать, что $f$ дифференцируемо по Гато, но не дифференцируемо по Фреше в т. $(0,0)$.
    
    \textbf{Решение.} 

    \begin{definition}
        Здесь и далее $X$ и $Y$ --- нормированные пространства.
        Отображение $F$ имеет вариацию по Лагранжу в точке $x_0$ если для любого $h \in X$ существует предел:
        \begin{equation*}
            \lim _{\lambda \rightarrow 0} \frac{F(x_0 + \lambda h) - F(x_0)}{\lambda} =: F'(x_0)[h].
        \end{equation*}
    \end{definition}

    \begin{definition}
        Отображение $F$ дифференцируемо по Гато в точке $x_0$, если оно имеет вариацию по Лагранжу в точке $x_0$,
        а отображение $F'(x_0)$ линейно и непрерывно.
    \end{definition}

    \begin{definition}
        Отображение $F$ дифференцируемо по Фреше в точке $x_0$, если существует линейный непрерывный оператор 
        $A: X \rightarrow Y$ такой, что $F(x_0 + h) = F(x_0) + A h + r(h)$, где отображение $r$ таково, что
        $\lim _{||h|| \rightarrow 0} \frac{||r(h)||_Y}{||h||_X} = 0 $.
    \end{definition}

    Вычислим вариацию по Лагранжу в нуле. Пусть $h \in \mathbb{R}^{2}$. Заметим, что прямая $\{t h: t \in \mathbb{R}\}$ пересекается с множеством $M$ не более, чем в одной точке. Значит, при малых $t$ выполнено $t h \notin M, F(t h)-F(0)=0$. Поэтому $F^{\prime}(0)[h]=0$. Это линейный непрерывный функционал. Значит, $F$ дифференцируемо по Гато в 0 . При этом $F$ в нуле разрывно и, следовательно, не дифференцируемо по Фреше.
    \end{task}
    