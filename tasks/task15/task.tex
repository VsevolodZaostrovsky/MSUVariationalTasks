\begin{task}
Привести пример гладких функций $f_{0}: \mathbb{R} \rightarrow \mathbb{R}, f_{1}: \mathbb{R} \rightarrow \mathbb{R}$ таких, что в задаче $f_{0}(x) 
\rightarrow \min , f_{1}(x)=0$ будет существовать точка локального минимума и в принципе Лагранжа будет $\lambda_{0}=0$ (а с $\lambda_{0} \neq 0$ принцип Лагранжа не выполнен).

Решение. Рассмотрим задачу

$$
x \rightarrow \inf , \quad x^{2}=0
$$

Единственная допустимая точка $-\hat{x}=0$. Значит, она и будет точкой минимума. Запишем функцию Лагранжа: $\mathcal{L}=\lambda_{0} x+\lambda_{1} x^{2}$. Приравнивая ее производную в $\hat{x}$ к нулю, получаем $\lambda_{0}=0$.
\end{task}