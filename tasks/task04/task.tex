\begin{task}
(задача о геодезических на плоскости Лобачевского.) Найти допустимые экстремали в задаче

$$
\int_{t_{0}}^{t_{1}} \frac{\sqrt{1+\dot{x}^{2}}}{x} d t \rightarrow \operatorname{extr}, x\left(t_{0}\right)=x_{0}, \quad x\left(t_{1}\right)=x_{1}, \quad x>0
$$

\textbf{Решение.} 
Имеем $L_{\dot{x}}=\frac{\dot{x}}{x \sqrt{1+\dot{x}^{2}}}, L_{\dot{x} \dot{x}}=\frac{1}{x\left(1+\dot{x}^{2}\right)^{3 / 2}}>0$. Значит, $\hat{x} \in$ $C^{2}\left[t_{0}, t_{1}\right]$ и $\dot{\hat{x}}(t) L_{\dot{x}}(\hat{x}(t), \dot{\hat{x}}(t))-L(\hat{x}(t), \dot{\hat{x}}(t))=$ const.

Проверим, что допустимая экстремаль не может обращаться в константу ни на каком невырожденном интервале. Это видно из уравнения Эйлера:

$$
-\frac{d}{d t} \frac{\dot{x}}{x \sqrt{1+\dot{x}^{2}}}-\frac{\sqrt{1+\dot{x}^{2}}}{x^{2}}=0
$$

(тогда бы получилось равенство $1 / \hat{x}^{2}(t) \equiv 0$ ).

Из уравнения $\dot{x} L_{\dot{x}}-L=$ const получаем

$$
\dot{x} \cdot \frac{\dot{x}}{x \sqrt{1+\dot{x}^{2}}}-\frac{\sqrt{1+\dot{x}^{2}}}{x}=\text { const. }
$$

Значит, $\frac{1}{x \sqrt{1+\dot{x}^{2}}}=$ const. Получаем $1+\dot{x}^{2}=\frac{c^{2}}{x^{2}}$, 
или $\dot{x}= \pm \sqrt{\frac{c^{2}}{x^{2}} - 1}$. На промежутках, где $\dot{x} \neq 0$, решаем это дифференциальное уравнение и получаем

$$
t-a= \pm \int \frac{d x}{\sqrt{\frac{c^{2}}{x^{2}} - 1}}= \pm \frac{1}{2} \int \frac{d x^{2}}{\sqrt{c^{2}-x^{2}}}= \pm \sqrt{c^{2}-x^{2}}
$$

Возводим в квадрат и получаем $x^{2}+(t-a)^{2}=c^{2}$. Это уравнение окружности с центром на горизонтальной оси.

Если в какой-то точке $\dot{x}$ обращается в 0 , то условия теоремы единственности нарушаются, но всё равно экстремаль задается уравнением окружности (склеивается из двух дуг окружностей; в силу гладкости обе дуги принадлежат одной и той же окружности; горизонтальных "вставок" быть не может, т.к. экстремаль не равна константе на интервалах).

Итак, геодезические - дуги окружности с центром на горизонтальной оси.

Утверждается, что найденная допустимая экстремаль будет точкой глобального минимума. В самом деле, $L_{\dot{x} \dot{x}}>0$ при $x>0$, так что $L$ выпукла по $\dot{x}$.

\end{task}