\begin{task}
Доказать, используя поле экстремалей, что допустимая экстремаль в задаче

$$
\int_{t_{0}}^{t_{1}} \frac{\sqrt{1+\dot{x}^{2}}}{\sqrt{x}} d t \rightarrow \operatorname{extr}, \quad x\left(t_{0}\right)=x_{0}, \quad x\left(t_{1}\right)=x_{1}, \quad x>0
$$

является точкой глобального минимума (здесь $x_{0}>0, x_{1}>0$ ).

\textbf{Peшение.} . Мы уже вычисляли экстремали в параметрическом виде:

$$
x=c(1-\cos \tau), \quad t-a=c(\tau-\sin \tau)
$$

где $a \in \mathbb{R}, c>0, \tau \in[0,2 \pi]$. Будем их обозначать $x(t, a, c)$.

Наша цель доказать: допустимая экстремаль будет точкой глобального минимума.

Сначала покажем, что $\dot{x}(t, a, c)$ строго убывает, при этом принимает все вещественные значения. В самом деле, $t(\tau)$ строго возрастает; $\frac{d x}{d t}=\frac{d x}{d \tau} \cdot \frac{d \tau}{d t}=\frac{\sin \tau}{1-\cos \tau} ;$ при $\tau \rightarrow+0$ предел равен $+\infty$, при $\tau \rightarrow 2 \pi-0$ предел $-\infty$. Производная по $\tau$ от $\frac{\sin \tau}{1-\cos \tau}$ равна $-\frac{1}{1-\cos \tau}<0$, так что $\ddot{x}<0$.
\end{task}