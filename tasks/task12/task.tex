\begin{task}
    Показать, что если отображение $F$ строго дифференцируемо в нуле и дифференцируемо по Гато в окрестности 0 , то производная по Гато $F_{G}(x): h \mapsto F_{G}(x)[h]$ непрерывна в 0 .\\
    \textbf{Решение.} \\
    Если $F$ строго дифференцируемо в 0 , т.е.
    
    $$
    \left\|F(x)-F(y)-F^{\prime}(0)(x-y)\right\| \leq \varepsilon\|x-y\| \quad \Longrightarrow \quad\left\|F(x)-F(y)-F_{G}^{\prime}(0)(x-y)\right\| \leq \varepsilon\|x-y\|
    $$
    
    при $\|x\| \leq \delta(\varepsilon)$ и $\|y\| \leq \delta(\varepsilon)$, то константа Липшица отображения $x \mapsto F(x)-F^{\prime}(0)[x]$, обозначенная здесь через $\varepsilon$ стремится к 0 в дельта-окрестности 0 при дельта стремящемся к 0 .
    
    Поскольку отображение $\Phi: h \mapsto\left(F_{G}^{\prime}(x)-F_{G}^{\prime}(0)\right)[h]$ липшицево с константой эпсилон в окрестности 0 , то в каждой точке этой окрестности норма производной $\Phi$ не больше эпсилон. Поэтому приходим к заключению: $\left\|F_{G}^{\prime}(x)-F_{G}^{\prime}(0)\right\|$ не больше эпсилон, если $x$ в маленькой окрестности нуля. Требуемая непрерывность установлена.
\end{task}