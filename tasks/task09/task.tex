\begin{task}
    Построить пример отображений $F:\mathbb{R}\rightarrow \mathbb{R}^2$, $G:\mathbb{R}^2\rightarrow \mathbb{R}$ таких, что $F$ дифференцируемо по Фреше в т. $0$, $G$ дифференцируемо по Гато в т. $(0,0)$, $F(0)=(0,0)$, при этом $G \circ F$ не имеет вариации по Лагранжу в т. 0.
    
    \textbf{Peшение.} 
    $$F:\mathbb{R}\rightarrow \mathbb{R}^2, \ \ \ x \rightarrow (x,x^2)$$
    \begin{equation*}
    G:\mathbb{R}^2\rightarrow \mathbb{R}, (x_1,x_2)\rightarrow
    \begin{cases}
        1, (x_1,x_2)\in M \\0, (x_1,x_2)\notin M
    \end{cases}
    \end{equation*}
    Где $M=\{ (x_1,x_2)|x_2=x_1,x_1>0\}$, уже знаем что G дифференцируемо по Гато в т. $(0,0)$ (но не по Фреше). Покажем, что $F$ дифференцируемо по Фреше в т. 0.
    $$\underset{\lambda \rightarrow 0}{lim}=\frac{F(\lambda h)-F (0)}{\lambda} 
        = \underset{\lambda\rightarrow0}{lim}=\frac{(\lambda h,\lambda^2 h^2)-(0,0)}{\lambda}=(h,0)=F'(0)[h]$$
    
    $$ F(\lambda h)=(\lambda h,\lambda^2 h^2)=(0,0)+(h,0)+((\lambda-1)h,\lambda^2 h^2) $$
    Последнее слогаемое это $\bar{o}(\|h \|)$.
    
    \noindent $\Rightarrow F$ дифференцируемо по Фреше в 0.$G \circ F:\mathbb{R}\rightarrow\mathbb{R}$:
     \begin{equation*}
    x\rightarrow
    \begin{cases}
        1, x\neq0\\0, x=0
    \end{cases}
    \end{equation*}
    $$\underset{\lambda\rightarrow0}{lim}=\frac{G \circ F(\lambda h)-G \circ F(0)}{\lambda}=\underset{\lambda\rightarrow0}{lim} \left( \frac{1}{\lambda} \right) $$
    $\Rightarrow$ нет вариации по Лагранжу.
    \end{task}