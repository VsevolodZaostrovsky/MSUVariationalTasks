\begin{task}

\textbf{Задача 8.}

Пусть $M=\left\{\left(x_{1}, x_{2}\right) \in \mathbb{R}^{2}: x_{1}>0, x_{2}=x_{1}^{2}\right\}, f: \mathbb{R}^{2} \rightarrow \mathbb{R}$,

$$
f\left(x_{1}, x_{2}\right)= \begin{cases}1, & \left(x_{1}, x_{2}\right) \in M \\ 0, & \left(x_{1}, x_{2}\right) \notin M\end{cases}
$$

Показать, что $f$ дифференцируемо по Гато, но не дифференцируемо по Фреше в т. $(0,0)$.\\

\textbf{Решение.} Вычислим вариацию по Лагранжу в нуле. Пусть $h \in \mathbb{R}^{2}$. Заметим, что прямая $\{t h: t \in \mathbb{R}\}$ пересекается с множеством $M$ не более, чем в одной точке. Значит, при малых $t$ выполнено $t h \notin M, F(t h)-F(0)=0$. Поэтому $F^{\prime}(0)[h]=0$. Это линейный непрерывный функционал. Значит, $F$ дифференцируемо по Гато в 0 . При этом $F$ в нуле разрывно и, следовательно, не дифференцируемо по Фреше.

\end{task}
