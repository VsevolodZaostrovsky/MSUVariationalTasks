\begin{task}
Привести пример такой задачи выпуклого программирования, что допустимая $\hat{x}-$ не есть точка минимума, но существует ненулевой набор $\left(\lambda_{0}, \ldots, \lambda_{m}\right)$, удовлетворяющий условиям а)-с) теоремы Куна-Таккера.

\begin{theorem}
    (Каруш - Кун - Таккер). Пусть $X$ --- линейное пространство, $f_0, \ldots, f_m: X \rightarrow \mathbb{R} \cup\{+\infty\}$ --- выпуклые функии.
    \begin{enumerate}
        \item (необходимое условие). Пусть $\hat{x}$ - точка минимума в задаче: 
                \\
                $\left\{\begin{array}{l}f_0(x) \rightarrow \inf \\ f_j(x) \leq 0, \quad 1 \leq j \leq m\end{array}\right.$
                \\

        Тогда существует ненулевой набор чисел $\lambda_0, \lambda_1, \ldots, \lambda_m$ со следующими свойствами: \\
            (a) $\lambda_j \geq 0,0 \leq j \leq m$ (условие неотрицательности); \\
            (b) $\lambda_j f_j(\hat{x})=0, 1 \leq j \leq m$ (условие дополняющей нежесткости); \\
            (c) $\hat{x}$ является точкой минимума функции $\mathcal{L}(x):=\sum_{j=0}^m \lambda_j f_j(x)$ (условие минимума). \\
        \item (достаточное условие). Пусть $\hat{x}-$ допустимая точка. Пусть существует набор чисел $\lambda_0, \lambda_1, \ldots, \lambda_m$ 
            со свойствами а)-с), при этом $\lambda_0>0$. Тогда $\hat{x}-$ точка минимума в рассматриваемой задаче.
        \item Пусть существует точка $\bar{x} \in X$ такая, что $f_j(\bar{x})<0,1 \leq j \leq m$ (условие Слейтера). 
        Тогда, если $\lambda_0, \lambda_1, \ldots, \lambda_m$ - ненулевой набор чисел со свойствами а)-с), то $\lambda_0>0$.
    \end{enumerate}
\end{theorem}

\textbf{Пример.}
% Если $\hat{x}$ - решение задачи на минимум $f_{0}(x)$ при условии $f_{1}(\hat{x})=0, f_{2}(\hat{x})=0$, где функционалы выпуклы, 
% то для функции Лагранжа $\mathcal{L}(x)=$ $\sum_{j \geq 0} \lambda_{j} f_{j}(x)$ справедливы условия

% a) минимум функции Лагранжа достигается на решении;

% b) $\lambda_{j} f_{j}(\hat{x})=0, j \geq 1$;

% c) $\lambda_{j} \geq 0, j \geq 0$.

Пусть $x=\left(x_{1}, x_{2}\right) \in \mathbb{R}^{2}, f_{1}(x)=-x_{1}, f_{2}(x)=x_{2}$, а 
$f_{0}(x)=x_{1}^{2}+\left(x_{2}+1\right)^{2}$. 
Тогда для функции Лагранжа ${\mathcal{L}(x)=-x_1}$ имеем: точка $\hat{x}=(0,0)$ 
- допустимая, условия а)-с) выполнены, но минимум $f_{0}(x)$ достигается в точке $(0,-1)$.


\end{task}
