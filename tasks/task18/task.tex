\begin{task}
Привести пример такой задачи выпуклого программирования, что допустимая $\hat{x}-$ не есть точка минимума, но существует ненулевой набор $\left(\lambda_{0}, \ldots, \lambda_{m}\right)$, удовлетворяющий условиям а)-с) теоремы Куна-Таккера.

\textbf{Пример.}
Если $\hat{x}$ - решение задачи на минимум $f_{0}(x)$ при условии $f_{1}(\hat{x})=0, f_{2}(\hat{x})=0$, где функционалы выпуклы, то для функции Лагранжа $\mathcal{L}(x)=$ $\sum_{j \geq 0} \lambda_{j} f_{j}(x)$ справедливы условия

a) минимум функции Лагранжа достигается на решении;

b) $\lambda_{j} f_{j}(\hat{x})=0, j \geq 1$;

c) $\lambda_{j} \geq 0, j \geq 0$.

Пусть $x=\left(x_{1}, x_{2}\right) \in \mathbb{R}^{2}, f_{1}(x)=x_{1}, f_{2}(x)=x_{2}$, а $f_{0}(x)=x_{1}^{2}+\left(x_{2}-1\right)^{2}$. Тогда для функции Лагранжа $\mathcal{L}(x)=f_{1}(x)+f_{2}(x)$ имеем: точка $\hat{x}=(0,0)$ - допустимая, условия а)-с) выполнены, но минимум $f_{0}(x)$ достигается в точке $(0,1)$.


\end{task}
