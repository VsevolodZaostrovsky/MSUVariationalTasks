\begin{task}
    \begin{definition}
        Скажем, что выполнено усиленное условие Лежандра,
        если $\widehat L_{\dot{x}\dot{x}} > 0 \; \forall t \in [-T_0, T_0]$.
    \end{definition}

    \begin{definition} Скажем, что выполнено условие Якоби,
        если справедливо усиленное условие Лежандра, а решение уравнения Якоби
        \begin{equation} \label{eqJacobi28}
            -\frac{d}{d t}\left(\widehat{L}_{\dot{x} \dot{x}}(t) \dot{h}
            +\widehat{L}_{\dot{x} x}(t) h\right)
            +\left(\widehat{L}_{\dot{x} x}(t) \dot{h}
            +\widehat{L}_{x x}(t) h\right)=0 \quad
            \Leftrightarrow \quad \frac{d}{d t}\left(\widehat{L}_{\dot{x} \dot{x}}(t) \dot{h}\right)
            =\left(\widehat{L}_{x x}(t)-\frac{d}{d t} \widehat{L}_{\dot{x} x}(t)\right) h
        \end{equation}
        не обращается в ноль на интервале $\left(-T_0, T_0\right)$ при начальных условиях:
        $h\left(-T_0\right)=0, \quad \dot{h}\left(-T_0\right)=1$.
    \end{definition}
    \begin{definition}
        Скажем, что выполнено усиленное условие Якоби, если справедливо усиленное условие
        Лежандра, а решение уравнения \eqref{eqJacobi28} не обращается в ноль на полусегменте
        $(-T_0, T_0]$ при начальных условиях: ${h(-T_0)=0, \; \dot{h}(-T_0)=1}$.
    \end{definition}

    \begin{definition}
        Скажем, что выполнено усиленное условие Вейерштрасса,
        если функиия $\dot{x} \mapsto L(t, x(t), \dot{x})$
        выпукла в '$C$'-окрестности экстремали $\widehat{x}$
        при любом $t \in\left[-T_0, T_0\right]$, т.е. для любого
        $t \in\left[-T_0, T_0\right]$ u $x(t) \in
            \mathcal{O}(\widehat{x}, \varepsilon) \;
            (\text {с некоторым } {\varepsilon>0})$ функиия
        $\dot{x} \mapsto L(t, x(t), \dot{x})$ выпукла.
    \end{definition}

    \begin{theorem}
        Если выполнены усиленное
        условие Якоби и усиленное условие Вейерштрасса, то экстремаль доставляет
        сильный максимум.
    \end{theorem}

    Рассмотрим задачу $\int_0^\pi\left(\dot{x}^2-x^2-x^4\right) d t \rightarrow$ inf, $x(0)=x(\pi)=$ 0.
    Показать, что для $\hat{x}=0$ выполнено yсиленное условие Лежандра, условие Якоби,
    при этом $\hat{x}=0$ не является точкой слабого минимума.

    \textbf{Peшение.} Имеем $\hat{L}_{\dot{x} \dot{x}}(t)=2, \hat{L}_{\dot{x} x}=0, \hat{L}_{x x}=-2$. 3начит, выполнено усиленное условие Лежандра. Уравнение Якоби имеет вид $\ddot{h}+h=0$; его нетривиальное решение, зануляющееся при $t=0$, имеет вид $h(t)=a \sin t$, $a \neq 0$. Torда $h(t) \neq 0$ при $t \in(0, \pi)$, но $h(\pi)=0$. Значит, выполнено условие Якоби, но не усиленное.
    Boзьмем $x(t)=\varepsilon \sin t$. Toгдa
    $$
        \begin{gathered}
            \int_0^\pi\left(\dot{x}^2-x^2-x^4\right) d t=\varepsilon^2 \int_0^\pi\left(\cos ^2 t-\sin ^2 t\right) d t-\varepsilon^4 \int_0^\pi \sin ^4 t d t= \\
            =\varepsilon^2 \int_0^\pi \cos 2 t d t-\varepsilon^4 \int_0^\pi \sin ^4 t d t=-\varepsilon^4 \int_0^\pi \sin ^4 t d t<0 .
        \end{gathered}
    $$

    Так как $\varepsilon>0$ может быть сколь угодно мало, то слабого минимума нет.
\end{task}