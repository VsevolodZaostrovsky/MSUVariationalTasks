\begin{task} \label{task6}
    Найти допустимые экстремали в задаче
    \[  \int_{-T_{0}}^{T_{0}} x\sqrt{1 + \dot{x}} \,dt \rightarrow extr\text{, }x(-T_{0}) = x(T_{0}) = \xi\text{, } x > 0 \]
    В зависимости от $\xi > 0$ установить, сколько может быть допустимых экстремалей. \\
    \textbf{Решение.} \\
    \[L_{\dot{x}} = \frac{x\dot{x}}{\sqrt{1 + \dot{x}^2}}\]
    \[L_{\dot{x}\dot{x}} = {(\frac{x}{(1 + \dot{x}^{2})})^ {3/2}} > 0,\]
    Решение уравнения Эйлера $\hat{x} \in C^2[-T_{0}, T_{0}]$ и $\hat{x}(t)\dot{L_{x}}(\hat{x}(t),
        \dot{\hat{x}}(t)) - L(\hat{x}(t),
        \dot{\hat{x}}(t)) = const$.\\
    Проверим, что допустимая экстремаль не может обращаться в константу ни на каком невырожденном отрезке. Это видно из уравнения Эйлера:
    \[-\frac{d}{dt} \frac{x\dot{x}}{\sqrt{1 + \dot{x}^2}} + \sqrt{1 + \dot{x}^2} = 0\]
    Решений нет. \\
    Из уравнения $\dot{x}L_{\dot{x}} - L = const$ получаем: \\
    \[ \dot{x} \frac{x\dot{x}}{\sqrt{1 + \dot{x}^2}} - x\sqrt{1 + \dot{x}^2} = const\]
    \[\frac{x}{\sqrt{1 + \dot{x}^2}} = const = C \Rightarrow 1 + \dot{x}^2 = \frac{x^2}{C^2} \Rightarrow \dot{x} = \pm \sqrt{\frac{x^2}{C^2} - 1} \Rightarrow dt = \pm \frac{dx}{\sqrt{\frac{x^2}{C^2} - 1}}\]
    \[t + a = \pm \int_{}^{}\frac{1}{\sqrt{\frac{x^2}{C^2} - 1}} \,dx = \left| x = Cch(\tau) \right| = \pm \int_{}^{}\frac{C sh(\tau)}{|sh(\tau)|} \,d\tau \]
    \[t + a = \pm C\tau \Rightarrow \frac{t+a}{C} = \pm \tau \Rightarrow ch\left(\frac{t+a}{C}\right) = ch(\tau) = \frac{x}{C} \Rightarrow x = Cch\left(\frac{t+a}{C}\right)\]
    \[x(-T_{0}) = x(T_{0}) \Rightarrow x = Cch\left(\frac{t}{C}\right)\]
    $C$ --- решение уравнения $Cch\left(\frac{T_{0}}{C}\right) = \xi$. Пусть $b = \frac{1}{C}$, тогда $ch(T_{0}b) = \xi b$. $ch(T_{0}b)$ --- выпуклая вниз функция, а $\xi b$ --- линейная $\Rightarrow$ может быть 0, 1 или 2 решениия.
    \[\xi_{*} > 0 \text{ --- случай касания.}\]
    \[\xi > \xi_{*} \text{ --- два решения.}\]
    \[\xi < \xi_{*} \text{ --- нет решений.}\]
    $ch(T_{0}b) = \xi_{*}b$, $T_{0}sh(T_{0}b) = \xi_{*}$. Исключаем $\xi_{*}$.
    \[ch(T_{0}b) = T_{0}bsh(T_{0}b)\]
    Отсюда находим $b$, а потом $\xi_{*}$.
\end{task}