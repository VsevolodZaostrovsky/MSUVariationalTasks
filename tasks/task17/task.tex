\begin{task}
Пусть $l>0$. Доказать, что допустимые экстремали в задаче

$$
\int_{0}^{1}(y \dot{x}-x \dot{y}) d t \rightarrow \max , \quad \int_{0}^{1} \sqrt{\dot{x}^{2}+\dot{y}^{2}} d t=l, \quad x(0)=x(1)=y(0)=y(1)=0, \quad \dot{x}^{2}+\dot{y}^{2}>0
$$

являются параметризацией окружности.

\textbf{Решение.} Функция Лагранжа имеет вид

$$
\int_{0}^{1}\left(\lambda_{0}(-y \dot{x}+x \dot{y})+\lambda_{1} \sqrt{\dot{x}^{2}+\dot{y}^{2}}\right) d t
$$

Значит, уравнения Эйлера имеют вид

$$
\begin{aligned}
& -\frac{d}{d t}\left(\lambda_{1} \frac{\dot{x}}{\sqrt{\dot{x}^{2}+\dot{y}^{2}}}-\lambda_{0} y\right)+\lambda_{0} \dot{y}=0, \\
& -\frac{d}{d t}\left(\lambda_{1} \frac{\dot{y}}{\sqrt{\dot{x}^{2}+\dot{y}^{2}}}+\lambda_{0} x\right)-\lambda_{0} \dot{x}=0 .
\end{aligned}
$$

Если $\lambda_{1}=0$, то $\lambda_{0} \dot{y}=0, \lambda_{0} \dot{x}=0$. Так как $\lambda_{0} \neq 0$, то $\dot{y}=0, \dot{x}=0$, что противоречит условию $\dot{x}^{2}+\dot{y}^{2}>0$.

Пусть $\lambda_{1} \neq 0$. Можно считать, что $\lambda_{1}=2$. Тогда

$$
-\frac{d}{d t} \frac{\dot{x}}{\sqrt{\dot{x}^{2}+\dot{y}^{2}}}+\lambda_{0} \dot{y}=0
$$

Пусть $\lambda_{1} \neq 0$. Можно считать, что $\lambda_{1}=2$. Тогда

$$
-\frac{d}{d t} \frac{\dot{x}}{\sqrt{\dot{x}^{2}+\dot{y}^{2}}}+\lambda_{0} \dot{y}=0
$$

$$
-\frac{d}{d t} \frac{\dot{y}}{\sqrt{\dot{x}^{2}+\dot{y}^{2}}}-\lambda_{0} \dot{x}=0
$$

откуда

$$
\frac{\dot{x}}{\sqrt{\dot{x}^{2}+\dot{y}^{2}}}=\lambda_{0} y+a, \quad \frac{\dot{y}}{\sqrt{\dot{x}^{2}+\dot{y}^{2}}}=-\lambda_{0} x+b
$$

Возведем обе части равенств в квадрат и получим

$$
1=\left(-\lambda_{0} y+a\right)^{2}+\left(\lambda_{0} x+b\right)^{2}
$$

Заметим, что $\lambda_{0} \neq 0$, иначе $\frac{d y}{d x}=$ const или $\frac{d x}{d y}=$ const, при этом $(\dot{x}, \dot{y})$ нигде не обращается в $(0,0)$. Тогда будет движение по отрезку всё время в одном направлении, что противоречит граничным условиям. А если $\lambda_{0} \neq 0$, то (3) - уравнение окружности.
\end{task}