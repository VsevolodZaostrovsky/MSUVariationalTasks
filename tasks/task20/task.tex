\begin{task}
    (аэродинамическая задача Ньютона). Найти допустимые экстремали в задаче
    $$
        \begin{cases}
            \int_0^{T_0} \frac{t}{1+u^2} d t \rightarrow \inf
            \\
            x(0)=0, x\left(T_0\right)=\xi
            \\
            \dot{x}=u
            \\
            u \geq 0
        \end{cases}
    $$

    здесь $T_0>0, \xi>0$ - заданные параметры. (Ответ для $\hat{x}(t)$ записывается в параметрическом виде: $x=x(v), t=t(v)$.)
    Доказать, что допустимая экстремаль существует и единственна, и что она будет точкой глобального минимума в данной задаче.

    \textbf{Решение.} \\
    Составим функцию Лагранжа:
    $$
        \mathcal{L}=\lambda_0 \int_0^{T_0} \frac{t}{1+u^2} d t+\int_0^{T_0} p(t)(\dot{x}-u) d t+\lambda_1 x(0)+\lambda_2\left(x\left(T_0\right)-\xi\right) .
    $$

    Необходимые условия локального минимума имеют вид $\lambda_0 \geq 0$ (условие неотрицательности), 
    $\dot{p}=0$ (уравнение Эйлера), $p(0)=\lambda_1, p\left(T_0\right)=-\lambda_2$ (условие трансверсальности),
    $$
        \min _{v \geq 0}\left(\frac{\lambda_0 t}{1+v^2}-p(t) v\right)=\frac{\lambda_0 t}{1+\hat{u}(t)^2}-p(t) \hat{u}(t)
    $$
    (принцип максимума Понтрягина).

    Из уравнения Әйлера получаем, что $p(t)=c$.
    Пусть $\lambda_0=0$. Тогда $\min _{v \geq 0}(-c v)=-c \hat{u}(t)$. Если $c=0$, то $p(t) \equiv 0$; 
    из условия трансверсальности следует, что $\lambda_1=\lambda_2=0$, то есть все множители Лагранжа нулевые. 
    Если $c>0$, то у функции $-c v$ на $[0,+\infty$ ) точки минимума нет. Если $c<0$, то $\hat{u}(t) \equiv 0$. 
    В силу граничного условия в нуле, $\hat{x}(t) \equiv 0$, что прогиворечиг с условием $x\left(T_0\right)=\xi>0$.

    Пусть $\lambda_0>0$. Без ограничения общности можно взять $\lambda_0=1$. Также обозначим $q=-c$. Получаем
    $$
        \min _{v \geq 0}\left(\frac{t}{1+v^2}+q v\right)=\frac{t}{1+\hat{u}(t)^2}+q \hat{u}(t) .
    $$

    Для фиксированного $t \in\left[0, T_0\right]$ положим $f(v)=\frac{t}{1+v^2}+q v$. 
    Если $q \leq 0$, то минимум функции $f$ не достигается, так как она строго убывает. Значит, остается случай $q>0$.
    Найдем участки монотонности функции $f$ на $\mathbb{R}_{+}$.
    Имеем: $f^{\prime}(v)=-\frac{2 t v}{\left(1+v^2\right)^2}+q$. Условие $f^{\prime}(v)=0$ эквивалентно уравнению
    \begin{equation} \label{eq201}
        q\left(1+v^2\right)^2-2 t v=0.
    \end{equation}

    В левой части стоит строго выпуклая функция, поэтому у нее количество нулей не больше 2. Также заметим, что максимальный корень строго возрастает по $t$. 
    В самом деле, если 
    ${ t_1<t_2$, $q\left(1+u_1^2\right)^2-2 t_1 u_1 = 0 }$, 
    то ${ q\left(1+u_1^2\right) ^2 - 2 t_2 u_1<0 } $. 
    Кроме того, максимальный корень стремится к $+\infty$ при $t \rightarrow+\infty$.

    Игак, либо $f$ строго возрастаег (тогда 0 гочка минимума), либо сначала возрастаег, погом убывает и затем снова возрастает. 
    Во втором случае минимум либо при $v=0$, либо в точке $u_*$, являющейся максимальным корнем уравнения \eqref{eq201}. 
    Сравним значения $f(0)$ и $f\left(u_*\right)$. Запишем неравенство $f(0) \leq f\left(u_*\right)$, получим $\frac{t u_*}{1+u_*^2} \leq q$; 
    подставим из \eqref{eq201} $t=\frac{q\left(1+u_*^2\right)^2}{2 u_*}$ и получим после вычислений $u_* \leq 1$.

    Итак, если $f$ строго возрастает или $u_*<1$, то минимум достигается в 0 ; если $u_*>1$, то минимум достигается в $u_*$. Заметим, что $u_*=1$ при $t=2 q$. 
    Так как $u_*$ строго возрастает по $t$, то при $t<2 q$ минимум функции $f$ достигается в 0 , а при $t>2 q \quad$ в $u_*$.
    Итак,
    $$
        \hat{u}(t)=\begin{cases} 
            0, \quad t<2 q, 
            \\
            u_*(t), \quad t>2 q .
        \end{cases}
    $$

    В силу условия $x(0)=0$, при $0 \leq t \leq 2 q$ получаем $\hat{x}(t)=0$.
    При $t \geq 2 q$ функцию $x(t)$ запишем параметрически. Выражая $t$ через $v$ из (2), получаем $t(v)=\frac{q}{2}\left(\frac{1}{v}+2 v+v^3\right)$. Далее, $\frac{d x}{d v}=\frac{d x}{d t} \cdot \frac{d t}{d v}=v \cdot \frac{q}{2}\left(-\frac{1}{v^2}+2+3 v^2\right)=\frac{q}{2}\left(-\frac{1}{v}+2 v+3 v^3\right)$. Значит, $x(v)=\frac{q}{2}\left(-\ln v+v^2+\frac{3}{4} v^4\right)+C$; константа $C$ находится из условия $x(1)=0$ (здесь мы вос пользовались тем, что $x$ непрерывна по $t$ и $u_*(2 q)=1$ ), т.е. $x(v)=\frac{q}{2}\left(-\ln v+v^2+\frac{3}{4} v^4-\frac{7}{4}\right)$. Игак,
    $$
        t(v)=\frac{q}{2}\left(\frac{1}{v}+2 v+v^3\right), \quad x(v)=\frac{q}{2}\left(-\ln v+v^2+\frac{3}{4} v^4-\frac{7}{4}\right) .
    $$

    Теперь покажем, что для любых $T_0>0, \xi>0$ найдется $q>0$ такое, что $x\left(T_0\right)=\xi$.
    Пусть $a=\frac{\xi}{T_0}$. Возьмем $q=2$. Покажем, что если $x_0(t)=0$ при $t \leq 2 q=4$, а при $t>4$ задано парамегрически :
    $$
        t(v)=\frac{1}{v}+2 v+v^3, \quad x_0(v)=-\ln v+v^2+\frac{3}{4} v^4-\frac{7}{4},
    $$

    то найдется такое $t_*>4$, что $x_0\left(t_*\right)=a t_*$. Затем определим $q$ из равенства $\frac{q t_*}{2}=T_0$.
    Мы уже говорили, что $\dot{x}_0(t)$ строго возрастает при $t>4$ и $\dot{x}_0(t) \underset{t \rightarrow \infty}{\rightarrow}+\infty$. 
    Значит, $x_0(t)-$ at $\underset{t \rightarrow+\infty}{\rightarrow}+\infty$. Kроме того, $x_0(4)-4 a<0$. 
    Поэтому уравнение $x_0(t)=a t$ имеет корень на $(4,+\infty)$. Единственность корня следует из строгой выпуклости $x_0$ на $[4,+\infty)$.

    Теперь покажем, что найденная экстремаль является точкой минимума. Достаточно показать, что она является точкой минимума функции $\mathcal{L}$. 
    Из условий трансверсальности следует, что $\lambda_1=-q, \lambda_2=q$. Имеем: $\mathcal{L}(x, u)=\mathcal{L}_1(x)+\mathcal{L}_2(u)$, где
    $$
    \mathcal{L}_1(x)=-\int_0^{T_0} q \dot{x} d t-q x(0)+q\left(x\left(T_0\right)-\xi\right) \equiv-q \xi T_0, 
    \quad \mathcal{L}_2(u)=\int_0^{T_0}\left(\frac{t}{1+u^2}+q u\right) d t .
    $$

    Из принципа максимума Понтрягина следует, что $\mathcal{L}_2(u) \geq \mathcal{L}_2(\hat{u})$. 
    Значит, $\mathcal{L}(x, u) \geq \mathcal{L}(\hat{x}, \hat{u})$ для любой допустимой пары $(x, u)$.

\end{task}